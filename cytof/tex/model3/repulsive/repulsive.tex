\documentclass[12pt]{article}
\usepackage{fullpage,amsmath}
\usepackage{setspace}
\usepackage{graphicx, psfrag, amsfonts}
\usepackage{natbib}


\newcommand{\Dt}{\mbox{$\tilde{D}$}}
\newcommand{\bu}{\mbox{\boldmath $u$}}

\newcommand{\bp}{\mbox{\boldmath $p$}}
\newcommand{\bq}{\mbox{\boldmath $q$}}
\newcommand{\bz}{\mbox{\boldmath $z$}}


\newcommand{\bzero}{\mbox{\boldmath $0$}}

\newcommand{\bT}{\mbox{\boldmath $T$}}
\newcommand{\bY}{\mbox{\boldmath $Y$}}
\newcommand{\bZ}{\mbox{\boldmath $Z$}}
\newcommand{\bd}{\mbox{\boldmath $d$}}
\newcommand{\D}{\mbox{${\cal D}$}}
\newcommand{\cH}{\mbox{${\cal H}$}}
\newcommand{\Q}{\mbox{${\cal Q}$}}
\newcommand{\cX}{\mbox{${\cal X}$}}
\newcommand{\cA}{\mbox{${\cal A}$}}

\def\bpsi{\mbox{\boldmath $\psi$}}
\def\bal{\mbox{\boldmath $\alpha$}}
\def\bdel{\mbox{\boldmath $\delta$}}
\def\bmu{\mbox{\boldmath $\mu$}}
\def\bxi{\mbox{\boldmath $\xi$}}
\def\bet{\mbox{\boldmath $\eta$}}
\def\bth{\mbox{\boldmath $\theta$}}
\newcommand{\xibar}{\bar{\xi}}
\newcommand{\etbar}{\bar{\eta}}
\newcommand{\bebar}{\bar{\beta}}
\def\bbet{\mbox{\boldmath $\beta$}}
\newcommand{\bbebar}{\bar{\bbe}}


\newcommand{\Exp}{\mbox{E}}

\newcommand{\Bern}{\mbox{Bern}}
\newcommand{\Nor}{\mbox{N}}
\newcommand{\Ga}{\mbox{Gamma}}
\newcommand{\Dir}{\mbox{Dir}}
\newcommand{\Ber}{\mbox{Ber}}
\newcommand{\Be}{\mbox{Be}}
\newcommand{\Unif}{\mbox{Unif}}
\newcommand{\Rep}{\mbox{Rep}}

\newcommand{\iid}{\stackrel{iid}{\sim}}
\newcommand{\indep}{\stackrel{indep}{\sim}}


\begin{document}
\onehalfspacing

\noindent
\begin{center}
{\bf Repulsive Feature Allocation Model}
\end{center}

\vskip .1in
\today

\vskip .1in
Similar to the construction of the conventional IBP, we assume $v_k \iid \Be(\alpha/K, 1)$ with fixed $K > 1$.  We consider a prior for $\alpha$, $\alpha \sim \Ga(a_\alpha, b_\alpha)$.  We utilize a repulsive distribution in \cite{quinlan2017parsimonious} and develop a repulsive feature allocation model (rep-FAM) for $J \times K$ binary matrix $\bZ=[z_{ik}], j=1, \ldots, J$ and $k=1, \ldots, K$.  The rep-FAM assigns larger probabilities for $\bZ$ whose columns are more distinct.  Recall that columns $\bz_k$, $k=1, \ldots, K$ characterize latent cell phenotypes in CyTof application.  In particular, consider the following for $\bZ=[\bz_1, \ldots, \bz_K]$,
\begin{eqnarray*}
\bZ \sim \Rep_{J,K}(f_0, C_0, r),
\end{eqnarray*}
where
\begin{eqnarray}
f_{0k}(\bz_k) = \prod_{j=1}^J (v_k)^{z_{jk}} (1-v_k)^{(1-z_{jk})}, \nonumber \\
C_0(r) = \exp \left( - \frac{r^2}{2\nu} \right), \mbox{ with fixed } \nu>0 \label{eq:C_0} \\
r_{k_1, k_2} = \sqrt{\sum_{j=1}^J |z_{jk_1} - z_{jk_2}|}. \label{eq:diff}
\end{eqnarray}
$r_{k_1, k_2} \geq 0$ in \eqref{eq:diff} measures difference between columns $\bz_{k_1}$ and $\bz_{k_2}$.  $C_0(r)$ is a function decreasing in $r$ with $0 < C_0(r) \leq 1$.  We write the density function of $\bZ$ as follows; for given $J$ and $K$,
\begin{eqnarray}
p(\bZ \mid f_0, C_0, r, v_k) &\propto& \prod_{k=1}^K f_{0k}(\bz_k) \prod_{k_1=1}^{K-1} \prod_{k_2 = k_1+1}^K \left\{ 1- C_0(r_{k_1, k_2})\right\} \nonumber \\
&\propto&
\prod_{k=1}^K \prod_{j=1}^J (v_k)^{z_{jk}} (1-v_k)^{(1-z_{jk})} \nonumber \\
& &
\qquad 
\times
\underbrace{
\prod_{k_1=1}^{K-1} \prod_{k_2 = k_1+1}^K \left\{ 1- \exp \left( - \frac{\sum_{j=1}^J |z_{jk_1} - z_{jk_2}|}{2\nu} \right)\right\}.}
_\text{(A)} \label{eq:den_Z}
\end{eqnarray}
$p(\bZ \mid f_0, C_0, r, v_k)$ in \eqref{eq:den_Z} gives probability 0 to any $\bZ$ that has some repeated columns.  If $\bz_k$, $k=1, \ldots, K$ are very different from each other, $(A)$ in \eqref{eq:den_Z} is close to 1 and gives a larger probability than any $\bZ$ that has similar columns.  That is, the rep-FAM in \eqref{eq:den_Z} encourages latent cell phenotypes to have different expression combinations.  Other functions are used for $(A)$ in \eqref{eq:den_Z} in \cite{xie2017bayesian, petralia2012repulsive}.   Under the rep-FAM $\mbox{P}(z_{jk}=1) \neq v_k$ and depends on $\nu$, $K$, $J$ as well as $v_k$, unlike the conventional IBP.  We cannot analytically find $\mbox{P}(z_{jk}=1)$ since $\sum_{\bz_{K+1}} \Rep_{J, K+1} \neq \Rep_{J,K}$.   We numerically examine some properties of the rep-FAM through simulations and compare to those of the IBP.  




\bibliographystyle{natbib}
\bibliography{repulsive}


\end{document}
