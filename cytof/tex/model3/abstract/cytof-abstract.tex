\documentclass[12pt]{article}
\usepackage{fullpage,amsmath}
\usepackage{setspace}
\usepackage{graphicx, psfrag, amsfonts}


\newcommand{\Dt}{\mbox{$\tilde{D}$}}
\newcommand{\bu}{\mbox{\boldmath $u$}}

\newcommand{\bp}{\mbox{\boldmath $p$}}
\newcommand{\bq}{\mbox{\boldmath $q$}}
\newcommand{\bz}{\mbox{\boldmath $z$}}


\newcommand{\bzero}{\mbox{\boldmath $0$}}

\newcommand{\bT}{\mbox{\boldmath $T$}}
\newcommand{\bY}{\mbox{\boldmath $Y$}}
\newcommand{\bZ}{\mbox{\boldmath $Z$}}
\newcommand{\bd}{\mbox{\boldmath $d$}}
\newcommand{\D}{\mbox{${\cal D}$}}
\newcommand{\cH}{\mbox{${\cal H}$}}
\newcommand{\Q}{\mbox{${\cal Q}$}}
\newcommand{\cX}{\mbox{${\cal X}$}}
\newcommand{\cA}{\mbox{${\cal A}$}}

\def\bpsi{\mbox{\boldmath $\psi$}}
\def\bal{\mbox{\boldmath $\alpha$}}
\def\bdel{\mbox{\boldmath $\delta$}}
\def\bmu{\mbox{\boldmath $\mu$}}
\def\bxi{\mbox{\boldmath $\xi$}}
\def\bet{\mbox{\boldmath $\eta$}}
\def\bth{\mbox{\boldmath $\theta$}}
\newcommand{\xibar}{\bar{\xi}}
\newcommand{\etbar}{\bar{\eta}}
\newcommand{\bebar}{\bar{\beta}}
\def\bbet{\mbox{\boldmath $\beta$}}
\newcommand{\bbebar}{\bar{\bbe}}


\newcommand{\Exp}{\mbox{E}}

\newcommand{\Bern}{\mbox{Bern}}
\newcommand{\Nor}{\mbox{N}}
\newcommand{\Ga}{\mbox{Gamma}}
\newcommand{\Dir}{\mbox{Dir}}
\newcommand{\Ber}{\mbox{Ber}}
\newcommand{\Be}{\mbox{Be}}
\newcommand{\Unif}{\mbox{Unif}}

\newcommand{\iid}{\stackrel{iid}{\sim}}
\newcommand{\indep}{\stackrel{indep}{\sim}}


\begin{document}
\onehalfspacing

\noindent
\today

\vspace*{0.2in}
\noindent
{\bf Title:} Bayesian Feature Allocation Models for Natural Killer Cell Repertoire Studies Using Mass Cytometry Data

\vspace*{0.2in}
\noindent
{\bf Abstract:}  Bayesian feature allocation models (FAMs) embedded with
clustering are developed to analyze mass cytometry data, with primary aim to
characterize underlying cell repertoire structures.   Cell repertoires in
samples are heterogeneous. Each repertoire consists of a collection of cells
possessing different phenotypes that can be characterized by differences in
expression levels of cell surface markers.  In particular, mass cytometry data
collected to study the clinical efficacy of natural killer (NK) cells as
immunotherapeutic agents against leukemia is considered. NK cells play a
critical role in cancer immune surveillance and are the first line of defense
against viruses and transformed tumor cells.  The data includes expression
levels of 32 surface markers on each of thousands of cells from multiple
samples. NK cell repertoires may affect both NK cell function and immune
surveillance.  A key conceptual shift compared with existing approaches is to
explicitly characterize latent cell phenotypes through a FAM.  The models
simultaneously (1) characterize NK cell phenotypes based on expression /
non-expression of surface markers, (2) estimate compositions of the samples
based on the identified phenotypes, and (3) infer associations of subjects'
covariates, with the composition of the identified phenotypes in the samples.
The conventional Indian buffet process (IBP), one of the most popular feature
allocation models, is first utilized to model cell phenotypes. Non-ignorable
missing data that is present due to technical artifacts in mass cytometry are
accounted for using an informed prior missing mechanism. The repulsive FAM
(rep-FAM) is next proposed.  In contrast with the IBP, the rep-FAM produces a
parsimonious representation of phenotypes by discouraging the creation of
redundant phenotypes, and thus can improve inference on phenotypes.  Further
extensions to incorporate subject-based covariates are discussed to provide
inferences on phenotypes potentially associated with positive clinical
outcomes.  

\end{document}
