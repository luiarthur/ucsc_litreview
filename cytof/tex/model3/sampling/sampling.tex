\documentclass[12pt,]{article}

%{{{1
\usepackage{lmodern}
\usepackage{amssymb,amsmath}
%\usepackage{ifxetex,ifluatex}
%\usepackage{fixltx2e} % provides \textsubscript
%\ifnum 0\ifxetex 1\fi\ifluatex 1\fi=0 % if pdftex
%  \usepackage[T1]{fontenc}
%  \usepackage[utf8]{inputenc}
%\else % if luatex or xelatex
%  \ifxetex
%    \usepackage{mathspec}
%  \else
%    \usepackage{fontspec}
%  \fi
%  \defaultfontfeatures{Ligatures=TeX,Scale=MatchLowercase}
%\fi
% use upquote if available, for straight quotes in verbatim environments
%\IfFileExists{upquote.sty}{\usepackage{upquote}}{}
%% use microtype if available
%\IfFileExists{microtype.sty}{%
%\usepackage{microtype}
%\UseMicrotypeSet[protrusion]{basicmath} % disable protrusion for tt fonts
%}{}
\usepackage{setspace}
\usepackage[margin=1in]{geometry}
\usepackage[unicode=true]{hyperref}
\hypersetup{
            pdftitle={Sampling Scheme for CYTOF Model3},
            pdfauthor={Arthur Lui},
            pdfborder={0 0 0},
            breaklinks=true}
\urlstyle{same}  % don't use monospace font for urls
\usepackage{natbib}
\bibliographystyle{plainnat}
\IfFileExists{parskip.sty}{%
\usepackage{parskip}
}{% else
\setlength{\parindent}{0pt}
\setlength{\parskip}{6pt plus 2pt minus 1pt}
}
\setlength{\emergencystretch}{3em}  % prevent overfull lines
\providecommand{\tightlist}{%
  \setlength{\itemsep}{0pt}\setlength{\parskip}{0pt}}
\setcounter{secnumdepth}{5}
% Redefines (sub)paragraphs to behave more like sections
\ifx\paragraph\undefined\else
\let\oldparagraph\paragraph
%\renewcommand{\paragraph}[1]{\oldparagraph{#1}\mbox{}}
\fi
\ifx\subparagraph\undefined\else
\let\oldsubparagraph\subparagraph
\renewcommand{\subparagraph}[1]{\oldsubparagraph{#1}\mbox{}}
\fi


\usepackage{bm}
\usepackage{bbm}
\usepackage{graphicx}
\newcommand{\norm}[1]{\left\lVert#1\right\rVert}
\newcommand{\p}[1]{\left(#1\right)}
\newcommand{\bk}[1]{\left[#1\right]}
\newcommand{\bc}[1]{ \left\{#1\right\} }
\newcommand{\abs}[1]{ \left|#1\right| }
\newcommand{\mat}{ \begin{pmatrix} }
\newcommand{\tam}{ \end{pmatrix} }
\newcommand{\suml}{ \sum_{i=1}^n }
\newcommand{\prodl}{ \prod_{i=1}^n }
\newcommand{\ds}{ \displaystyle }
\newcommand{\df}[2]{ \frac{d#1}{d#2} }
\newcommand{\ddf}[2]{ \frac{d^2#1}{d{#2}^2} }
\newcommand{\pd}[2]{ \frac{\partial#1}{\partial#2} }
\newcommand{\pdd}[2]{\frac{\partial^2#1}{\partial{#2}^2} }
\newcommand{\N}{ \mathcal{N} }
\newcommand{\E}{ \text{E} }
\def\given{~\bigg|~}
\usepackage{float}
\def\beginmyfig{\begin{figure}[H]\center}
\def\endmyfig{\end{figure}}
\newcommand{\iid}{\overset{iid}{\sim}}
\newcommand{\ind}{\overset{ind}{\sim}}
\newcommand{\I}{\mathrm{\mathbf{I}}}

\def\bet{\bm{\eta}}


\allowdisplaybreaks
\def\M{\mathcal{M}}
\def\logit{\text{logit}}
\def\Bern{\text{Bernoulli}}
\def\N{\text{Normal}}
\def\G{\text{Gamma}}
\def\IG{\text{Inverse-Gamma}}
\def\Dir{\text{Dirichlet}}
\def\Be{\text{Beta}}
\def\lin{\lambda_{in}}
\def\btheta{\bm{\theta}}
\def\y{\bm{y}}
\newcommand\m{\bm{m}}
\def\mus{\mu^*}
\def\sss{{\sigma^2}^*}
\newcommand{\mhSpiel}[1]{
Since the full conditional distribution cannot be directly sampled from, it may
be sampled from by a Metropolis step with a Normal proposal distribution. 
The proposed state is accepted with probability 
$$
\min\bc{1, \frac{p(\tilde #1 \mid \y, \rest)}{p(#1\mid \y, \rest)}}.
$$
}


\newcommand{\mhLogSpiel}[2]{
Since the full conditional distribution for $#1$ cannot be directly sampled
from, it may be sampled from by a Metropolis step with a Normal proposal
distribution.  The parameter first needs to be log-transformed.  
Let the full conditional of the transformed parameter be 
$p(#2 \mid \y, \rest) = p_{#1}(\exp(#2) \mid \y, \rest) \exp(#2)$.
Then, the proposed state of the transformed parameter ($#2$) is accepted with
probability
$$
\min\bc{1, 
  \frac{p(\tilde #2  \mid \y, \rest)}
       {p(#2 \mid \y, \rest)}
}.
$$
Exponentiating the updated value for $#2$ returns the updated value for $#1$.
}


\newcommand{\mhLogitSpiel}[2]{
Since the full conditional distribution for $#1$ cannot be directly sampled
from, it may be sampled from by a Metropolis step with a Normal proposal
distribution.  The parameter first needs to be logit-transformed. 
Let the full conditional of the transformed parameter be 
$p(#2 \mid \y, \rest) = p_{#1}\p{\frac{1}{1+\exp(-#2)} \mid \y, \rest} 
\frac{\exp(-#2)}{(1 + \exp(- #2))^2}$.
The proposed
state of the transformed parameter ($#2$) is accepted with probability 
$$
\min\bc{1, 
  \frac{p(\tilde #2  \mid \y, \rest)}
       {p(#2 \mid \y, \rest)}
}
$$
Taking the inverse-logit of the updated value for $#2$ returns the updated
value for $#1$.
}



\newcommand{\Ind}[1]{\mathbbm{1}\bc{#1}}
\def\rest{\text{rest}}
\def\bang{\boldsymbol{\cdot}}
\def\h{\bm{h}}
\def\Z{\bm{Z}}
\def\Unif{\text{Unif}}

%sim-tex-commands
\newcommand{\true}{{\mbox{\tiny TR}}}
\newcommand{\bZ}{\mbox{\boldmath $Z$}}
\newcommand{\bp}{\mbox{\boldmath $p$}}
\newcommand{\bq}{\mbox{\boldmath $q$}}
\newcommand{\bz}{\mbox{\boldmath $z$}}
\newcommand{\bw}{\mbox{\boldmath $w$}}

%%% Graphing
\def\beginmyfig{\begin{figure}[H]\center}
\def\endmyfig{\end{figure}}
%}}}1

\title{Bayesian Feature Allocation Models for Natural Killer Cell Repertoire Studies Using Mass Cytometry Data}
\author{Arthur Lui}
\date{\today}

\begin{document}
\maketitle
\onehalfspacing



\begin{abstract}
\noindent
Bayesian feature allocation models (FAMs) embedded with
clustering are developed to analyze mass cytometry data, with primary aim to
characterize underlying cell repertoire structures.   Cell repertoires in
samples are heterogeneous. Each repertoire consists of a collection of cells
possessing different phenotypes that can be characterized by differences in
expression levels of cell surface markers.  In particular, mass cytometry data
collected to study the clinical efficacy of natural killer (NK) cells as
immunotherapeutic agents against leukemia is considered. NK cells play a
critical role in cancer immune surveillance and are the first line of defense
against viruses and transformed tumor cells.  The data includes expression
levels of 32 surface markers on each of thousands of cells from multiple
samples. NK cell repertoires may affect both NK cell function and immune
surveillance.  A key conceptual shift compared with existing approaches is to
explicitly characterize latent cell phenotypes through a FAM.  The models
simultaneously (1) characterize NK cell phenotypes based on expression /
non-expression of surface markers, (2) estimate compositions of the samples
based on the identified phenotypes, and (3) infer associations of subjects'
covariates, with the composition of the identified phenotypes in the samples.
The conventional Indian buffet process (IBP), one of the most popular feature
allocation models, is first utilized to model cell phenotypes. Non-ignorable
missing data that is present due to technical artifacts in mass cytometry are
accounted for using an informed prior missing mechanism. The repulsive FAM
(rep-FAM) is next proposed.  In contrast with the IBP, the rep-FAM produces a
parsimonious representation of phenotypes by discouraging the creation of
redundant phenotypes, and thus can improve inference on phenotypes.  Further
extensions to incorporate subject-based covariates are discussed to provide
inferences on phenotypes potentially associated with positive clinical
outcomes.  

%\noindent
%{\em Keywords:} ~  Count data, Laplace prior, Metagenomics, Microbiome, Regularizing prior, Process convolution,  Negative binomial model, Next-generation sequencing
\end{abstract}




\section{Introduction}
Clinical application of natural killer (NK) cells %as immunotherapeutic agents against leukemia
recently has emerged as a powerful treatment modality for advanced cancers refractory to conventional therapies \citep{rezvani2015application}. % and intensely investigated.
%NK cells, the third lymphocyte lineage, % preceded by T cells and B cells,
%play critical roles in the immune response to certain %virus infected cells and
%transformed tumor cells.
NK cells play a critical role in cancer immune surveillance and are the first line of defense against viruses and transformed tumor cells.
They have the intrinsic ability to infiltrate cancer tissue and their presence in the tumor is reported to be associated with better clinical outcomes \citep{suck2016natural}.  %NK cells develop in the bone marrow and are ``educated'' during their development to ensure not to attack normal self cells whiling rapidly killing tumor or virally infected cells.
% (called self-tolerance).
%During the development,
Drs. Thall and Rezvani, collaborators at UT MD Anderson Cancer Center, have conducted clinical trials to study potential clinical efficacy of umbilical cord blood (UCB) transplantation as a therapy for leukemia.  UCB has become an established source of hematopoietic stem cells for transplantation. UCB NK cell therapy has the advantages of low risk of viral transmission from donor to recipient \citep{sarvaria2017umbilical}. In the trials, leukemia patients received UCB cell transplants. % after irradiation therapy.  %Some characteristics of the patients such as their survival after the treatment are recorded.
During follow up, samples were taken at multiple time points from each patient.  Samples from %In addition, peripheral blood samples from
healthy subjects and cord blood samples also were collected for comparison to leukemia patient samples.
The samples were processed and expressions of 32 NK cell-associated cell surface protein markers, %in individual cells of the samples
measured for individual cells in the samples using mass cytometry.
Their primary research goal is to understand phenotypes and functions structured across heterogeneous NK cells. % based on the markers' expression levels.
Better understanding of the characteristics of NK cells is crucial to estimate the true potential of NK cell therapies against cancer.  %for cancer immunotherapy. %against leukemia and possibly also against solid tumors.



%NK cells play critical roles in defending against tumors. Furthermore, their diversity and function are
%known to be linked. Researchers have thus studied NK cell diversity from
%various perspectives. For instance, it is known that NK cell diversity
%is lower at birth \citep{strauss2015human} than in adults. Some
%researchers have studied the effect of introducing diverse NK cells into
%tumor patients. Yet again, some researchers have found that patients
%with higher NK diversity are associated with higher exposure risk of
%HIV-1, suggesting that existing diversity may decrease flexibility of
%the antiviral response. Many questions about NK cells remain to be
%answered. Understanding NK diversity through spectrometry has therefore
%been an important research area in the bio-sciences.   

%The main
%inferential goal of this project is to identify the NK cell phenotypes
%(or cell-types) in various samples as a set of subpopulations of the set
%of some provided surface markers. The NK cell-types are latent, and for
%\(J\) markers \(2^J\) different cell-types can be considered. This
%provides a computational challenge when the number of markers is even
%moderately large. Thirty-two markers are included in this analysis, and
%naively enumerating all possible markers is not feasible. We therefore,
%use a latent feature allocation model to learn the latent structure of
%predominant cell-types. Latent feature models have been successfully
%applied to various problems and will be reviewed in the following
%section.
%Data for this project is rendered through CyTOF analyses of
%NK-cell-targeting markers. Having some understanding of CyTOF and NK
%cells their importance is therefore necessary.


Advances in cytometry has led to more research and greater understanding  of
natural killer (NK) cells and how their diversity impacts immunity against the
development of tumors and other viral diseases. Flow cytometry (developed by
Wallace Coulter in the 1950's) is a laser-based biophysical technology which is
sometimes used for biomarker detection. It is regularly used to diagnose health
disorders like cancer. Cytometry has advanced over the years.
Fluorescence-based flow cytometry, which makes use of fluorescent dyes and
lasers that emit light at specific wavelengths, is one such advancement that
has been mainstream for several decades \citep{herzenberg2002history}.  In recent
years, a new technique called Cytometry at time-of-flight (CyTOF) has surfaced.
It makes use of time-of-flight mass spectrometry, where sophisticated devices
are used to accelerate, separate, and identify ions by mass. This new method
warrants the analysis of multiple parameters in shorter time \citep{cheung2011screening}. 
%Through CyTOF, scientists have been able to better understand natural killer
%(NK) cells \citep{horowitz2013genetic}. 
A major challenge in deciphering mass cytometry data is to develop efficient
inferential frameworks that can handle the complexity of the data, including
high dimensionality and noise. Many existing computational methods use
dimension reduction techniques and/or clustering-based approaches. 
%{\tt introduce some existing approaches.  Explain a bit the methods
%that we will use for comparison.  State limitations of existing methods.  Are
%they Bayesian?  Are they modeling phenotypes like our $\Z$.  Emphasize that our
%methods produce direct inferences on phenotypes or quantification of
%uncertainty associated with their inference.}
FlowSOM \citep{van2015flowsom} is one such method which uses a self-organizing
map to cluster flow cytometry data. While among existing methods it is
considered the fastest and most flexible at finding subpopulations, it is not
able to measure uncertainty about the learned clusters.
To overcome the limitations of existing methods, we propose to develop novel
Bayesian feature allocation models (FAMs) embedded with clustering. 



\subsection{Literature Review: Feature Allocation Models}\label{literature-review}
One of the main inferential goals is  to learn a latent structure of predominant cell phenotypes, where cell phenotype are composed of distinct expression combinations of the markers. Specifically, phenotype $k$ is represented by a $J$-dim binary vector, $\bm z_k=(z_{1k}, \ldots, z_{Jk})$ where $z_{jk}$ take either of 1 or 0 if marker $j$ is or is not expressed in phenotype $k$.  For \(J\) markers, \(2^J\) phenotypes can be constructed.
One could create a \(J \times 2^J\) matrix, which contains all possible
phenotypes generated by the \(J\) markers, with each column representing 
each phenotype. It is computationally infeasible when \(J\) is large. Taking a Bayesian approach, we consider a prior probability model for binary matrices with infinite number of columns and learn the predominant ones in posterior that  generate observed data . %For computational efficiency, we require a flexible prior which will also learn the number of cell-types \((K)\). 
These models are also called latent feature allocation models.  One of popular models for binary feature matrices is the Indian buffet process (IBP) proposed by \citet{griffiths2011indian}. 
%The IBP have been used in a variety of applications where modelling latent binary features is of interest. 

\citet{griffiths2011indian} construct the IBP by considering the finite feature allocation
model and taking the limit with respect to the number of features; for a given $K$,
\begin{align}
\begin{split}
v_k \mid \alpha &\sim \text{Beta}(\alpha/K, 1),~ k=1, \ldots, K \\
z_{jk} \mid \pi_k &\sim \text{Bernoulli}(v_k),~ k=1, \ldots, K~\mbox{ and } j=1, \ldots, J. \\
\end{split}
\label{eq:ibp}
\end{align}
The marginal limiting distribution of $\Z$ defines an IBP as $K \rightarrow
\infty$ and dropping all columns with all 0s, that is,  \(Z \sim
\text{IBP}(\alpha)\). Under the IBP, each row is expected to have  \(\alpha\)
active features.  The number of columns with $\sum_{j=1}^J z_{jk} >0$ is random
and it can be shown that the number of such columns follows
\(\text{Poisson}(\alpha \sum_{j=1}^J j^{-1})\).   A prior distribution can be
placed on \(\alpha\) to reflect uncertainty. A gamma prior is popular for
$\alpha$ due to conjugacy.  \citet{teh2007stick} represented the IBP using the
stick-breaking construction similar to the stick-breaking representation of the
Dirichlet process (DP).  \citet{williamson2010dependent} developed a dependent
IBP (dIBP) to induce correlations between objects
(rows) and to model multiple $\Z$s that can be dependent.



{\tt need to include Tamara's papers on feature allocation model, my papers on tumor heterogeneity. 
\begin{itemize}
\item ``Feature allocations, probability functions, and paintboxes.'', ``Combinatorial clustering and the beta negative binomial process.'' etc by Tamara \url{http://www.tamarabroderick.com/papers.html}

\item ``Bayesian Inference for Intra-Tumor Heterogeneity in Mutations and Copy Number Variation'', ``A Bayesian Feature Allocation Model for Tumor Heterogeneity.'' by me

\item ``Bay- clone: Bayesian nonparametric inference of tumor subclones using ngs data'' by Subhajit et al.
 
\item YX's works: ``Bayesian Inference for Latent Biologic Structure with Determinantal Point Processes'', "MAD Bayes for Tumor Heterogeneity Feature Allocation with Exponential Family Sampling"  \url{http://www.ams.jhu.edu/~yxu70/pub.html}.

\item There could be more.  Please search for those.
\end{itemize}

}

\section{Project 1: }
\subsection{Introduction}
% {\tt short intro for project 1}
In this project, we analyze marker expression data from a CyTOF analysis of
blood samples from patients. Each sample consists of tens of thousands of cells
and records expression levels for 32 NK-cell markers. High marker expression
levels correspond to the expression of that marker, while low marker expression
levels and missing values correspond to non-expression. For a given cell, the
expression of certain markers corresponds to a phenotype. We are interested in
identifying phenotypes that occur frequently within each sample. This is
important to practitioners because NK-cell diversity is known to be affect
immunity against infectious diseases. We therefore propose modelling marker
expression levels with a flexible mixture model, and the latent cell-type
structure with a latent feature allocation model.

\subsection{Model}\label{prob-model}
\subsubsection{Sampling Model} 
Samples are taken from \(I\) subjects, \(i = 1,2,...,I\). Sample \(i\)
consists of \(N_i\) cells, \(n=1, \ldots, N_i\) and for each cell,
expression levels of \(J\) markers are measured. Let
\(\tilde{y}_{inj} \in \mathbb{R}^+\) represent the raw measurement of an
expression level of marker \(j\) of cell \(n\) in sample \(i\). Let
\(c_{ij}\) denote the ``cutoff'' for
 marker \(j\) in sample \(i\). A marker of a cell is likely to be expressed if its observed expression level is greater than the cutoff. A value of $\tilde{y}_{inj}$ below the cutoff may imply that marker $j$ is not expressed in cell $n$ of sample $i$.    We consider the logarithm transformation
after scaling \(\tilde{y}_{inj}\) by \(c_{ij}\), \[
y_{inj}=\log\p{\frac{\tilde{y}_{inj}}{c_{ij}}} \in \mathbb{R}.
\] For some \((i, n, j)\), \(\tilde{y}_{inj}\) is missing and we
introduce a binary indicator, \[
m_{inj} = \begin{cases}
  0, & \text{if $\tilde{y}_{inj}$ is observed,} \\
  1, & \text{if $\tilde{y}_{inj}$ is missing.}
\end{cases}
\] That is, \(m_{inj}=1\) indicates that the expression level of marker
\(j\) of cell \(n\) in sample \(i\) is missing.

%\begin{enumerate}
\def\labelenumi{\arabic{enumi}.}
%\tightlist
%\item
%  The data have infinite support.
%\item
%  \(y_{inj} = 0\) has a special meaning, which is that the data take on
%  the same value as the cutoff. Consequently, \(y_{inj} > 0\) means that
%  the data take on values greater than the cutoff, etc.
%\item
%  \(y_{inj}\) for which \(\tilde y_{inj} = 0\) are regarded as missing,
%  and is to be imputed.
%\end{enumerate}

%\newpage


We assume that a sample has heterogeneous cells having $K$ different phenotypes.  The phenotypes are not directly observable and we introduce latent phenotype indicators \(\lambda_{in} \in \{1, \ldots, K\}\), for cell $n$ in sample $i$, \(i=1, \ldots, I\) and \(n=1, \ldots, N_i\).
%\(\lambda_{in} \in \{1, \ldots, K\}\) denotes the cell phenotype of cell \(n\) in sample \(i\) defined by 
The cell phenotypes are defined by columns of \(J \times K\) binary matrix \(\Z\). Element \(z_{j, k} \in \{0, 1\}\) indicates if marker \(j\) is expressed in cell phenotype \(k\). Event \(z_{jk}=0\) represents that marker \(j\) is not expressed for phenotype $k$, and \(z_{jk}=1\) for
expression. We let $\Z$ and $\lambda_{in}$ random. Details will be discussed later.  Given \(z_{j, \lambda_{in}} \in \{0, 1\}\), we assume a mixture of
normals for \(y_{inj}\),
\begin{align}
y_{inj} \mid \eta_{ij}, \mu^\star, \sigma^{2 \star}_{i} \ind
\begin{cases}
\sum_{\ell=1}^{L^0} \eta^0_{ij\ell}~ \N(\mu^\star_{0\ell}, \sigma^{2 \star}_{0i\ell}), &\mbox{if $z_{j,\lambda_{in}}=0$},\\
\sum_{\ell=1}^{L^1} \eta^1_{ij\ell}~ \N(\mu^\star_{1\ell}, \sigma^{2 \star}_{1i\ell}), &\mbox{if $z_{j,\lambda_{in}}=1$},\\
\end{cases} \label{eq:y-mix}
\end{align}
where the number of mixture components \(L^0\) and \(L^1\) are fixed.  The vectors $\bet^0_{ij}$ and $\bet^1_{ij}$ are mixture weights with \(\sum_{\ell} \eta^0_{ij\ell}\sum_{\ell}= \eta^1_{ij\ell}=1\) where \(0 < \eta^1_{ij\ell} < 1\) and \(0 < \eta^0_{ij\ell} < 1\).  In \eqref{eq:y-mix}, $\mu^\star$ are common for all samples and markers but $\sigma^{2, \star}$ are indexed by sample $i$ to account for sample specific variability.  The mixture model can thus flexibly capture various features in data.  For easy computation, we introduce mixture component indicators $\gamma_{inj}$ for $y_{inj}$.  Given \(\lambda_{in}=k\) we define
\(\gamma_{inj}\); for \(i=1, \ldots, I\), \(n=1, \ldots, N_i\) and
\(j=1, \ldots, J\),
\begin{eqnarray}
p(\gamma_{inj} = \ell)=\eta^{z_{jk}}_{ij\ell}, \mbox{ where }~ \ell \in \{1,\ldots, L^{z_{jk}}\}. \label{eq:gam}
\end{eqnarray}
Given \(\lambda_{in}=k\) and \(\gamma_{inj}=\ell\), we assume a normal
distribution for \(y_{inj}\); for \(i=1, \ldots, I\),
\(n=1, \ldots, N_i\) and \(j=1, \ldots, J\),
\begin{align}
  y_{inj} \mid \mu_{inj}, \sigma^2_{inj}  &\ind \N(\mu_{inj}, \sigma^2_{inj}), \label{eq:y-gam}
\end{align}
where \(\mu_{inj} = \mu^\star_{z_{j,k},\ell}\) and \(\sigma^2_{inj} =
{\sigma^{2}}^\star_{iz_{j,k}\ell}\). After marginalizing over $\gamma_{inj}$,
the model in \eqref{eq:y-gam} and \eqref{eq:gam} is equivalent to the model in
\eqref{eq:y-mix}.  

We next build a model for the missingness mechanism.
%{\tt explain our approach for missing in few sentences.}
This mechanism should reflect our belief that marker expression-levels are
recorded as ``missing'' when signals from cells for a marker are extremely
weak.  In addition, we will impose the restriction that extremely low
expression values have low prior probability. This ensures that parameters in
the sampling density will not be unnecessarily inflated.
%
Given
\(y_{inj}\), we consider a selection function for \(m_{inj}\); for
\(i=1, \ldots, I\), \(n=1, \ldots, N_i\) and \(j=1, \ldots, J\),
\begin{align}
  m_{inj} \mid p_{inj} &\ind \Bern(p_{inj}) \nonumber \\
  \logit(p_{inj}) &= \begin{cases}
  \beta_{0i} - \beta_{1i}(y_{inj}-c_0)^2, & \text{if } y_{inj} < c_0\nonumber, \\
  \beta_{0i} - \beta_{1i}c_1\p{y_{inj}-c_0}^{1/2}, & \text{otherwise}, \nonumber \\
  \end{cases} \label{eq:missing}
\end{align}
where \(c_0\) and \(c_1\) are real constants, $\beta_{0i} \in \mathbb{R}$ , and
$\beta_{1i} > 0$.
%{\tt say something like it is ``untestable'' but take this approach by
%incorporating our input from biologists}
Note that the assumptions for the distribution of the unobserved data are
untestable. However, we can incorporate input from biologists through informed
prior specifications. Prior uncertainty can then be directly propagated to the
posterior distribution for the missing mechanism.



\subsubsection{Priors}\label{priors}
\paragraph*{Latent cell phenotypes}  Recall that we characterize cell phenotypes with a $J\times K$ binary matrix \(\Z =\{z_{jk}\}\).  Following \citet{williamson2010dependent}, we assume
\begin{eqnarray*}
v_k \mid \alpha &\iid& \Be(\alpha/K, 1),~ k=1, \ldots, K, \\
\h_k &\iid& \N_J(\bm{0}, \Gamma), \\ 
z_{jk} \mid h_{jk}, v_k &=& \mathbb{I}\left\{ \Phi(h_{jk} \mid 0, \Gamma_{jj}) < v_k \right\},
\end{eqnarray*}
where $\Phi(h \mid m, s)$ is the cumulative distribution function of the normal
distribution with mean $m$ and variance $s$ and $\mathbb{I}(\cdot)$ is an
indicator function having 1 if $\Phi(h_{jk} \mid 0, \Gamma_{jj}) < v_k$ or 0
otherwise.  As $K \rightarrow \infty$, the limiting distribution of $Z$ is the
IBP {\tt cite the paper}.  Interactions between $J$ markers in phenotypes can
be modeled through $\bm G$.  Due to the multivariate probit construction for
$\Z$, $\Gamma$ is not identifiable and it is common to restrict $\Gamma$ to be
a correlation matrix.
%
{\tt discuss a bit models for correlation matrix.  I will send you some
references on models for correlation matrix.}
%
%
We let $\alpha \sim \G(a_\alpha, b_\alpha)$ with mean $a_\alpha/b_\alpha$.  

The $K$ cell phenotypes are common in all samples but the relative weights vary
across samples. Let $W_{ik}$ denote an abundance level of phenotype $k$ in
sample $i$.  We assume independent Dirichlet priors for $\bm W_i=(W_{i1},
\ldots, W_{iK})$ given $K$, $\bm W_{i} \mid K \iid \Dir_K(d/K)$. For latent
cell phenotype indicators, we let $p(\lin=k \mid \bm W_i) = W_{ik}$.

\paragraph*{Parameters in the Mixture for $y$}
In \eqref{eq:y-mix}, normal mixture models are assumed for $y_{inj}$. The mean
expression level of maker $j$ in cell $n$ is determined by its phenotype
$\lambda_{in}$.  In particular, if the marker is not expressed in the cell
type, $z_{j \lambda_{in}}=0$, its mean expression level is below the cutoff,
that is, a negative value.  If the marker is expressed $z_{j \lambda_{in}}=0$,
the expression of marker $j$ take a positive value.   Recall that
$\mus_{0\ell}$, $\ell=1, \ldots, L^0$ are mixture locations for $z_{j
\lambda_{in}}=0$ and $\mus_{1\ell}$, $\ell=1, \ldots, L^1$ for $z_{j
\lambda_{in}}=1$.  We assume 
\begin{align*}
\mus_{0\ell} \mid \psi_0, \tau^2_0 &\iid \N_-(\psi_0, \tau^2_0), ~~~ \ell \in \bc{1,...,L^0}, \\
\mus_{1\ell} \mid \psi_1, \tau^2_1 &\iid \N_+(\psi_1, \tau^2_1), ~~~ \ell \in \bc{1,...,L^1}, 
\end{align*}
where \(\N_-(m,s^2)\) and \(\N_+(m,s^2)\) denote the normal distribution with
mean \(m\) and variance \(s^2\), truncated to take on only negative values and
positive values, respectively.  The variances in the mixture components differ
by the value of $z_{j \lambda_{in}}$ and also vary across samples. We let, for
\(i=1, \ldots, I\),
\begin{align*}
\sigma^2_{0i\ell} \mid s_i &\ind \IG(a_\sigma, s_i), ~~~ \ell \in \bc{1,...,L^0}, \\
\sigma^2_{1i\ell} \mid s_i &\ind \IG(a_\sigma, s_i), ~~~ \ell \in \bc{1,...,L^1}.  
\end{align*}
We also assume $s_i \iid \G(a_s, b_s)$, $i \in \bc{1,...,I}$, with mean
\(a_s/b_s\). Lastly, we consider a model for mixture weights. To flexibly model
the distribution of $y$, we assume for a marker in a sample have two sets of it
own weights, one for each value of $z$, $\bm\eta^0_{ij}$ and $\bm\eta^1_{ij}$,
for each $(i, j)$. So for \(i=1, \ldots, I\), \(n=1, \ldots, N_i\) and \(j=1,
\ldots, J\),
\begin{align*}
\bm\eta^0_{ij} &\iid \Dir_{L^0}(a_{\eta^0}/L^0), \\
\bm\eta^1_{ij} &\iid \Dir_{L^1}(a_{\eta^1}/L^1). 
\end{align*}


\paragraph*{Parameters for Missingness Mechanism}
%{\tt explain what you do for missingness mechanism.}
%
A prior distribution over the missing mechanism can be specified through
placing priors on the parameters $\beta_{0i}$ and $\beta_{1i}$. 
%
We assume that $\beta_{0i} \iid \N(m_{\beta_0}, s^2_{\beta_0})$ and $\beta_{1i}
\iid \N_+(m_{\beta_1}, s^2_{\beta_1})$, $i=1, \ldots, I$.  We use data to
specify the values of the fixed hyperparameters, $m_{\beta_0}$ and
$m_{\beta_1}$. We let $s^2_{\beta_0}$ and $s^2_{\beta_1}$ be small to induce a
informative prior for $\beta_{0i}$ and $\beta_{1i}$. One way of determining
priors for the parameters in the missing mechanism is described in detail in 
the derivation of the full conditionals for $\beta$ in the appendix.



\subsubsection{Posterior Computation}\label{sampling-via-mcmc}
Let \(\btheta=\{\bm \mu^\star_0, \bm \mu^\star_1, \bm \sigma^2_{0i}, \bm \sigma^2_{1i}, \bm \eta^0, \bm \eta^1, \bm \lambda, \bm \gamma, \bm v, \bm h, \bm \beta_0, \bm \beta_1\}\) represent all random parameters.  Let \(\y\) and \(\m\) denote all \(y_{inj}\) and \(m_{inj}$ for all \((i,n,j)\), respectively. The joint posterior distribution is 
\begin{align*}
p(\btheta \mid \y, \m) &\propto 
p(\btheta) \prod_{i,n,j} p(m_{inj} \mid y_{inj}, \btheta) p(y_{inj} \mid \btheta) \nonumber\\
&=  
p(\btheta)
\prod_{i,n,j} \left[
  p_{inj}^{m_{inj}} (1-p_{inj})^{1-m_{inj}} \times 
   \frac{1}{\sqrt{2\pi\sigma^2_{inj}}} \exp\bc{-\frac{(y_{inj}-\mu_{inj})^2}{2\sigma^2_{inj}}}
\right].
\end{align*}
%The marginal density for \(y_{i,n,j}\) after integrating out
%\(\lambda\) and \(\gamma\) is
%\begin{align}
%p(y_{inj} \mid \btheta) = \sum_{k=1}^K W_{ik} \sum_{\ell=1}^{L^{Z_{jk}}}
%\eta^{Z_{jk}}_{ijl} \cdot \N(y_{inj} \mid \mu^\star_{Z_{jk}, \ell}, {\sigma^2}^\star_{i,Z_{jk},\ell}).
%\end{align}
Posterior simulation can be done via Gibbs sampling by repeatedly updating each
parameter one at a time until convergence. Parameter updates are made by
sampling from it full conditional distribution. Where this cannot be done
conveniently, a metropolis step can be used.
%
%{\tt do you use the marginal density of $y$ in posterior simulation?  If so,
%explain why we use it.}
%
Details of the posterior simulation are described in Appendix A. 


Summarizing the joint posterior distribution $p(\btheta \mid \y, \m)$ is
challenging, especially for $\Z$, $\bm W$ and $\lambda$, possibly due to a
label switching problem.
%
%{\tt Include David's approach here and explain his
%approach.  Also explain why we include $w$ in the adjacency matrix, which is
%different from David's approach. }
We use a method based on sequentially-allocated latent structure optimization
(SALSO) \citep{salso} to find point estimates for $\Z$, $\bm W$, and $\lambda$.
As a brief overview, in SALSO, a point estimate for a binary feature matrix is
obtained by finding a $\hat{Z}$ that minimizes the expression

\begin{eqnarray}
\text{argmin}_Z\sum_{j=1}^J\sum_{j'=1}^J(A(Z)_{rc} - \bar{A}_{j,j'})^2
\label{eq:salso}
\end{eqnarray}

where $A(Z)$ is the pairwise allocation matrix corresponding to a binary matrix
$Z$, and $\hat A$ is the pairwise allocation matrix averaged over all posterior
samples of $Z$. Thus in the current application, $A_{j,j'} = \sum_{k=1}^K
\Ind{Z_{j,k}=1}\Ind{Z_{j',k}=1}$ is the number of times that marker $j$ and
marker $j'$ share a feature. 
%
We adapted SALSO so that we can obtain point estimates for each sample $i$
using the draws from the joint posterior of all the parameters.  That is, using
posterior Monte Carlo samples we find posterior point estimates for each sample
$\hat{\Z}_i$, $\hat{\bm W}_i$ and $\hat{\lambda}_{in}$ as follows.
Suppose we obtain \(B\) posterior samples simulated from the posterior
distribution of \(\theta\). For each posterior sample of \(\Z\) and \(\bm
W_i\), we compute a $J \times J$ adjacency matrix, \(\bm A_i^{(b)}
=\{A^{(b)}_{i,j,j'}\}\), where 
\[
A^{(b)}_{i,j,j'} = \sum_{k=1}^K W^{(b)}_{i,k} 
\mathbb{I}\left( z^{(b)}_{j,k} = 1\right)
\mathbb{I}\left(z^{(b)}_{j',k} = 1\right), b \in \bc{1,...,B}.
\]
We then compute the mean adjacency matrix \(\bar A_i = \sum_{b=1}^B A_i^{(b)} /
B\).  We report a posterior point estimate of $\Z_i$ by choosing
\[
\hat{\bm Z}_i = \text{argmin}_{\bm Z} \sum_{j,j'} (A_{i,j,j'}^{(b)} - \bar A_{i,j,j'})^2).
\]
Conditioning on $\hat{\Z}_i$, we report posterior point estimates $\hat{\bm
W}_i$ and $\hat{\lambda}_{in}$. Note that in our adaptation, we weight
the adjacency counts by the sample-specific weight for each latent feature.
This allows us to put greater weight on cell-types that are more prevalent in
the samples, and less weight on cell-types that make up a small proportion in
the samples.


%%% sim-study-proj1 %%%
\subsection{Simulation} %%% Change name to Simulation Study?
In order to understand the strengths and limitations of the proposed model, a
simulation study was conducted. The study generated data that resemble the
cord-blood (CB) CyTOF data. This section details the data-simulation strategy
and the results of the analysis on the simulated data using the proposed model.

\subsubsection{Data Simulation}
The steps for simulating data is as follows.
First specify dimensions of the data through $I$ (number of samples), $J$
(number of markers), and $N_i$ (number of cells in sample $i$). Specify the
dimensions of the parameters through $K$ (true number of latent cell-types),
$L^0$ (number of mixture components in density for cells not expressing a
marker), and $L^1$ (number of mixture components in density for cells
expressing a marker). Then fix the latent cell-type matrix $\bZ^\true$
according to $J$ and $K$. Set values for $\sigma^{2, \true}_{0i\ell}$,
$\sigma^{2,\true}_{1i\ell}$, $\mu^{\star, \true}_{0\ell}$ and $\mu^{\star,
\true}_{1\ell}$.  This can be done using empirical values from CB data.
Simulate $\bw^\true_i$ from a Dirichlet distribution with parameters $a_1,
\ldots, a_K$. This ensures that $\sum_{k=1}^K \bw^\true_i = 1$.
Similarly, $\eta^\true_{0ij}$ and $\eta^\true_{1ij}$ can be simulated
from Dirichlet distributions. Note that $\eta^\true_{0ij}$ is an
$L^0$-dimensional vector, while $\eta^\true_{0ij}$ is an $L^1$-dimensional
vector. Using these parameters, simulate $y_{inj}$ according to the sampling
density. Finally, probabilistically set some of the $y_{inj}$ to be missing.
This can be done by first predetermining a certain percentage ($p\%$) of the
data to be missing. This percentage should should be less than $\sum_k
w^\true_{ik}(1-z^\true_{jk})$ so that truly expressed cells are unlikely to be
missing, and so as to be consistent with the simulated $W^\true$ and
$\bZ^\true$. Sample ($N_i\times p\%$) observations of $y_{inj}$ without
replacement with the probability of missing for $y$ being proportional to some
predetermined missing mechanism (possibly different from the one used in this
model). Care should be taken to ensure $y$ with positive values have almost
zero probability of being missing.

\textbf{Simulated $\bZ^\true$ and $W^\true$}

Following this scheme, we generated data comparable to the CB data.  In our
simulated data we assume 3 samples, each having $J=32$ markers, and each sample
containing tens of thousands of cells. This size resembles that of the CB data.
We then simulated $\bZ^\true$ and $W^\true$ with number of columns being $K=10$
to include a variety of cell-types. Figure \ref{fig:sim-Z} shows the simulated
$\bZ^\true$ with the latent features sorted by $W_i$ for each sample. 

\beginmyfig
\includegraphics{img/sim/Z_true_all.pdf}
\caption{True $\bZ^T$ with cell-types sorted by $W_i$ for each sample}
\label{fig:sim-Z}
\endmyfig
%reference this figure like so: \ref{fig:sim-Z}

The data $\y$ are shown in Figure \ref{fig:sim-Y}. The number of cells in each
sample are $N=(30000, 20000, 10000)$. Note that in the upper panel contains
marker expression data for sample 1 (left), sample 2 (middle), and sample 3
(right). The lower panel contains the same data but with the rows grouped by
the true cell-types ($\lambda$).  Red regions represent high expression levels,
whereas blue regions represent low expression levels. Thus, through this
visualization, we can observe the true latent cell-types that are generating
the data. Note that in this simulated data set, the white regions represent
missing data.  Three and four mixture components were used for non-expressed
and expressed marker expression levels respectively. 

\beginmyfig
\includegraphics[scale=.3]{img/sim/Y001.png}
\includegraphics[scale=.3]{img/sim/Y002.png}
\includegraphics[scale=.3]{img/sim/Y003.png} \\
\includegraphics[scale=.3]{img/sim/Ysorted001.png}
\includegraphics[scale=.3]{img/sim/Ysorted002.png}
\includegraphics[scale=.3]{img/sim/Ysorted003.png} 
\caption{The simulated marker expression data $y$. The upper panel contains
data for sample 1 (left), sample 2 (middle), and sample 3 (right). The lower
panel is the same data but with the rows (cells) grouped according to the true
cell-types ($\lambda$).}
\label{fig:sim-Y}
\endmyfig
%reference this figure like so: \ref{fig:sim-Y}

\subsubsection{Results}

Using MCMC and the sampling scheme mentioned above, two thousand samples from
the joint posterior distribution of all the parameters were collected after a
burn-in period of 1000 iterations. Figure \ref{fig:sim-post-Z} provides
representation of the data $y$ sorted by a point-estimate of their cell-types,
for each sample.  These figures are supplemented by a point-estimate of $\Z$
and $W$. The point estimates for $\bZ$ resemble $\bZ^\true$ and vary only rarely by
small permutations in less common cell-types. The cell-types in $\Z$ are sorted
by a point-estimate for $W_i$.  These also resemble $W_i^\true$ but vary
occasionally from the truth for smaller values of $W_i$. Only the cell-types
that make up the top 90\% of cells are included in Figure \ref{fig:sim-post-Z}.

\textbf{Posterior Estimate for $\bZ$ and $W$}
\beginmyfig
\includegraphics[scale=.3]{img/sim/YZ001.png}
\includegraphics[scale=.3]{img/sim/YZ002.png}
\includegraphics[scale=.3]{img/sim/YZ003.png}
\caption{$y$ sorted by a point-estimate of their cell-types ($\lambda$), for
each sample.  Point-estimates of $\Z$ (sorted by $W_i$) and $W_i$ are provided
for each sample below each $y_i$.}
\label{fig:sim-post-Z}
\endmyfig

\textbf{How to include this?} \\
\textbf{Posterior Predictive for positive $\y$} \\
\textbf{Posterior Prob $Z_{inj} == 0$ for missing $y_{inj}$} \\

%%% cb-proj1 %%%
\subsection{Cord Blood Data}
In this section we fit the proposed model to a real data set comprising
cord-blood samples from three patients ($I=3$), with sample sizes $(41474,
10454, 5177)$, for $J=32$ markers. 2000 samples from the posterior distribution
were obtained after a burn-in period of 1000 iterations.

%\textbf{Posterior Estimate for $\bZ$ and $W$}
%\beginmyfig
%\includegraphics[scale=.3]{img/sim/YZ001.png}
%\includegraphics[scale=.3]{img/sim/YZ002.png}
%\includegraphics[scale=.3]{img/sim/YZ003.png}
%\caption{$y$ sorted by a point-estimate of their cell-types ($\lambda$), for
%each sample.  Point-estimates of $\Z$ (sorted by $W_i$) and $W_i$ are provided
%for each sample below each $y_i$.}
%\label{fig:sim-post-Z}
%\endmyfig




\subsection{Conclusions}

\section{Project 2: Repulsive Feature Allocation Model}




\bibliography{litreview.bib}


\appendix
\section{Posterior Computation for Project 1}
{\tt Please write a short intro about the posterior sampling and reorganize the full conditionals }


To sample from a distribution which is otherwise difficult to sample
from, the Metropolis-Hastings algorithm can be used. This is
particularly useful when sampling from a full conditional distribution
of one of many parameters in an MCMC based sampling scheme (such as a
Gibbs sampler). Say \(B\) samples from a distribution with density
\(p(\theta)\) is desired, one can do the following:

\begin{enumerate}
\def\labelenumi{\arabic{enumi}.}
\tightlist
\item
  Provide an initial value for the sampler, e.g. \(\theta^{(0)}\).
\item
  Repeat the following steps for \(i = 1,...,B\).
\item
  Sample a new value \(\tilde\theta\) for \(\theta^{(i)}\) from a
  proposal distribution \(Q(\cdot \mid \theta^{(i-1)})\).

  \begin{itemize}
  \tightlist
  \item
    Let \(q(\tilde\theta \mid \theta)\) be the density of the proposal
    distribution.
  \end{itemize}
\item
  Compute the ``acceptance ratio'' to be \[
     \rho=
     \min\bc{1, \frac{p(\tilde\theta)}{p(\theta^{(i-1)})} \Big/ 
            \frac{q(\tilde\theta\mid\theta^{(i-1)})}
                 {q(\theta^{(i-1)}\mid\tilde\theta)}
        }
     \]
\item
  Set \[
     \theta^{(i)} := 
     \begin{cases}
     \tilde\theta &\text{with probability } \rho \\
     \theta^{(i-1)} &\text{with probability } 1-\rho.
     \end{cases}
     \]
\end{enumerate}

Note that in the case of a symmetric proposal distribution, the
acceptance ratio simplifies further to be
\(\frac{p(\tilde\theta)}{p(\theta^{(i-1)})}\).

The proposal distribution should be chosen to have the same support as
the parameter. Transforming parameters to have infinite support can,
therefore, be convenient as a Normal proposal distribution can be used.
Moreover, as previously mentioned, the use of symmetric proposal
distributions (such as the Normal distribution) can simplify the
computation of the acceptance ratio.

Some common parameter transformations are therefore presented here:

\begin{enumerate}
\def\labelenumi{\arabic{enumi}.}
\tightlist
\item
  For parameters bounded between \((0,1)\), a logit-transformation may
  be used. Specifically, if a random variable \(X\) with density
  \(f_X(x)\) has support in the unit interval, then
  \(Y=\logit(X)=\log\p{\frac{p}{1-p}}\) will have density
  \(f_Y(y) = f_X\p{\frac{1}{1+\exp(-y)}}\frac{e^{-y}}{(1+e^{-y})^{2}}\).
\item
  For parameters with support \((0,\infty)\), a log-transformation may
  be used. Specifically, if a random variable \(X\) with density
  \(f_X(x)\) has positive support, then \(Y = \log(X)\) has pdf
  \(f_Y(y) = f_X(e^y) e^y\).
\end{enumerate}


{\tt instead of subsections, we may do ``itemize'' for full conditionals.  I commented out the full conditionals temporarily since it gives some errors in compiling.  Please comment in as you revise. }



%\end{document}

\subsection{\texorpdfstring{Full Conditional for
\(\bm \beta\)}{Full Conditional for \textbackslash{}bm \textbackslash{}beta}}\label{full-conditional-for-bm-beta}

Define \(f_{inj}\) to be

\begin{align*}
f_{inj} &:= P(m_{inj} \mid p_{inj}, y_{inj}) \\
&= p_{inj}^{m_{inj}} (1-p_{inj})^{1 - m_{inj}} \\
&= \left(\frac{1}{1+e^{-x_{inj}}} \right)^{m_{inj}}\left(\frac{1}{1+e^{x_{inj}}} \right)^{1-m_{inj}},
\end{align*}

where

\begin{align*}
  x_{inj} &:= \begin{cases}
  \beta_{0i} - \beta_{1i}(y_{inj}-c_0)^2, & \text{if } y_{inj} < c_0\nonumber \\
  \beta_{0i} - \beta_{1i}c_1\sqrt{y_{inj}-c_0}, & \text{otherwise}, \nonumber \\
  \end{cases}\\
\end{align*}

where \(c_0\) and \(c_1\) are real constant, $\beta_{0i} \in \mathbb{R}$, and
$\beta_{1i} > 0$.
One way to determine a prior for the missing mechanism is to first select three
points to constrain the missing mechanism $(c_\text{low}, p_\text{low})$,
$(c_0, p_0)$, and $(c_\text{high}, p_\text{high})$, and then 
solve for $\beta_0$, $\beta_1$, and $c_1$. Figure \ref{fig:prob-miss-eg} shows
an example missing mechanism where $(c_\text{low}, p_\text{low}) = (-6,0.1)$,
$(c_0, p_0)=(-2,.99)$, and $(c_\text{high}, p_\text{high}) = (-1,0.01)$.
\beginmyfig
\includegraphics[scale=.5]{img/prob_miss_example.pdf}
\caption{Example missing mechanism. The blue points serve as guides in
determining a missing mechanism. Values for $\beta$ and $c$ can be solved for
through a system of equations.}
\label{fig:prob-miss-eg}
\endmyfig

Through simple algebra, we can solve for $\beta$ and $c_0$ as follows:
\begin{align*}
  \beta_0 &:= \logit(p_0) \\
  \beta_1 &:= \frac{\beta_0 - \logit(p_\text{low})}{(y_\text{low} - c_0)^2} \\
  c_1 &:= \frac{\beta_0 - \logit(p_\text{high})}{\beta_1 ~ \sqrt{y_\text{high} - c_0} }. \\
\end{align*}
Hence, specifying three critical points in the missing mechanism can guide
the construction of its prior distribution.


\subsubsection{\texorpdfstring{Full Conditional for
\(\beta_{0i}\)}{Full Conditional for \textbackslash{}beta\_\{0i\}}}\label{full-conditional-for-beta_0i}

Recall that \(\beta_{0i} \iid \N(m_{\beta_0},s^2_{\beta_0})\).

\begin{align*}
p(\beta_{0i} \mid \y, \rest) &\propto
p(\beta_{0i}) \times \prod_{n=1}^{N_i} \prod_{j=1}^J f_{inj} \\
%
&\propto \exp\bc{\frac{-(\beta_{0i}-m_{\beta_0})^2}{2s^2_{\beta_0}}} \prod_{n=1}^{N_i} \prod_{j=1}^J f_{inj} \\
\end{align*}

\mhSpiel{\beta_{0i}}

\subsubsection{\texorpdfstring{Full Conditional for
\(\beta_{1i}\)}{Full Conditional for \textbackslash{}beta\_\{1i\}}}\label{full-conditional-for-beta_1i}

Recall that $\beta_{1i}\ind \N^+(m_{\beta_1}, s^2_{\beta_1})$.
%
\begin{align*}
p(\beta_{1i} \mid \y, \rest) &\propto
p(\beta_{1i}) \times 
\prod_{n=1}^{N_i} \prod_{j=1}^J f_{inj} \\
%
&\propto \exp\bc{-\frac{(\beta_{1i} - m_{\beta_1})^2}{2s^2_{\beta_1}}}
\Ind{\beta_{1i} > 0}
\prod_{n=1}^{N_i} \prod_{j=1}^J f_{inj} \\
\end{align*}

\mhLogSpiel{\beta_{1i}}{\xi}

\subsection{\texorpdfstring{Full Conditional for Missing
\(\bm \y\)}{Full Conditional for Missing \textbackslash{}bm \textbackslash{}y}}\label{full-conditional-for-missing-bm-y}

\begin{align*}
p(y_{inj} \mid m_{inj}=1, \rest) &\propto
p(m_{inj} =1\mid y_{inj}, \rest) ~
p(y_{inj} \mid \rest) \\
%
%&\propto
%\exp\bc{\frac{-(y_{inj} - \mu_{inj})^2}{2\sigma^2_{inj}}}
%f_{inj} \\
&\propto
p_{inj} 
\sum_{\ell=1}^{L^{Z_{j\lin}}} \eta^{Z_{j\lin}}_{ij\ell} \cdot \N(y_{inj} \mid \mu^*_{Z_{jk}, \ell}, {\sigma^2}^\star_{i,Z_{j\lin},\ell}).
\end{align*}

\mhSpiel{y_{inj}}

Note that \(f_{inj}\) is a function of \(y_{inj}\) and should be
computed accordingly.

\subsection{\texorpdfstring{Full Conditional for
\(\bm\mu^\star\)}{Full Conditional for \textbackslash{}bm\textbackslash{}mu\^{}\star}}\label{full-conditional-for-bmmu}

For \(\mus_{0\ell}\)g, let
\(S_{0i\ell} = \bc{(i,n,j) : \p{Z_{j,\lin} = 0 ~\cap~ \gamma_{inj} = \ell}}\)g
and \(|S_{0i\ell}|\) the cardinality of \(S_{0i\ell}\).

\newcommand\musZeroPostvarDenom{
  \frac{1}{\tau^2_0} + \sum_{i=1}^I\frac{|S_{0i\ell}|}{{\sigma^2}^\star_{0i\ell}}
}
\newcommand\musZeroPostMeanNum{
  \frac{\psi_0}{\tau^2_0} + 
  \sum_{i=1}^I \sum_{S_{0i\ell}}  
  \frac{y_{inj}}{{\sigma^2}^\star_{0i\ell}}
}

\begin{align*}
p(\mus_{0\ell} \mid \y, \rest) &\propto 
p(\mus_{0\ell} \mid \psi_0, \tau^2_0) \times p(\y \mid \mus_{0\ell},\rest) \\
%
&\propto
\Ind{\mus_{0\ell}<0} \exp\bc{\frac{-(\mus_{0\ell} - \psi_0)^2}{2\tau^2_{0}}}
\prod_{i=1}^I\prod_{(i,n,j)\in S_{0i\ell}} \exp\bc{\frac{-(y_{inj} - \mus_{0\ell})^2}{2{\sigma^2}^\star_{i0\ell}}} \\
%
&\propto
\exp\bc{
  -\frac{(\mus_{0\ell})^2}{2}\p{\musZeroPostvarDenom} + 
  \mus_{0\ell}\p{\musZeroPostMeanNum}
} \\ 
& ~~~ \times \Ind{\mus_{0i\ell}<0} \\
\end{align*}

\[
\renewcommand\musZeroPostvarDenom{
  1 + \tau^2_0\sum_{i=1}^I(|S_{0i\ell}|/{\sigma^2}^\star_{0i\ell})
}
\renewcommand\musZeroPostMeanNum{
  \psi_0 + \tau^2_0 \sum_{i=1}^I\sum_{S_{0i\ell}} (y_{inj} / {\sigma^2}^\star_{0i\ell})
}
\therefore \mus_{0l} \mid \y, \rest \ind \N_-\p{
  \frac{\musZeroPostMeanNum}{\musZeroPostvarDenom},
  \frac{\tau^2_0}{\musZeroPostvarDenom}
}
\]

Similarly for \(\mus_{1\ell}\)g, let
\(S_{1\ell} = \bc{(i,n,j) : \p{Z_{j,\lin} = 1 ~\cap~ \gamma_{inj} = \ell}}\)g
and \(|S_{1i\ell}|\) the cardinality of \(S_{1i\ell}\).

\[
\newcommand\musOnePostvarDenom{
  1 + \tau^2_1 \sum_{i=1}^I (|S_{1i\ell}|/{\sigma^2}^\star_{1i\ell})
}
\newcommand\musOnePostMeanNum{
  \psi_1 + \tau^2_1 \sum_{i=1}^I \sum_{S_{1i\ell}} (y_{inj} / {\sigma^2}^\star_{1i\ell})
}
\therefore \mus_{1l} \mid \y, \rest \ind \N_+\p{
  \frac{\musOnePostMeanNum}{\musOnePostvarDenom},
  \frac{\tau^2_1}{\musOnePostvarDenom}
}
\]

\subsection{\texorpdfstring{Full Conditional for
\(\bm{{\sigma^2}}^*\)}{Full Conditional for \textbackslash{}bm\{\{\textbackslash{}sigma\^{}2\}\}\^{}*}}\label{full-conditional-for-bmsigma2}

Let
\(S_{0i\ell} = \bc{(i, n,j): Z_{j,\lin} = 0 ~\cap~ \gamma_{inj}=\ell}\),
\(i=1, \ldots, I\).

\begin{align*}
p(\sss_{0i\ell} \mid \y, \rest) &\propto p(\sss_{0i\ell} \mid s_i) \times p(\y \mid \sss_{0i\ell}, \rest) \\
&\propto (\sss_{0i\ell})^{-a_\sigma-1} \exp\bc{-\frac{s_i}{\sss_{0i\ell}}} 
\prod_{(i,n,j)\in S_{0i\ell}} \bc{
  \frac{1}{\sqrt{2\sss_{0i\ell}}}
  \exp\bc{\frac{-(y_{inj}-\mus_{0\ell})^2}{2\sss_{0i\ell}}}
} \\
&\propto (\sss_{0i\ell})^{-(a_\sigma + \frac{\abs{S_{0i\ell}}}{2})-1}
\exp\bc{\p{\frac{1}{\sss_{0i\ell}}}\p{s_i + \sum_{(i,n,j)\in S_{0i\ell}}
\frac{(y_{inj}-\mus_{0\ell})^2}{2}
}}.
\end{align*}

\[
\therefore \sss_{0i\ell} \mid \y, \rest \ind
\IG\p{a_\sigma + \frac{\abs{S_{0i\ell}}}{2}, ~~ s_i + \sum_{(i,n,j)\in S_{0i\ell}}
\frac{(y_{inj}-\mus_{0\ell})^2}{2}
}.
\]

Similarly, let
\(S_{1i\ell} = \bc{(i, n,j): Z_{j,\lin} = 1 ~\cap~ \gamma_{inj}=\ell}\).
Then, the full conditional for \(\sss_{1i\ell}\) is \[
\therefore \sss_{1i\ell} \mid \y, \rest \ind
\IG\p{a_\sigma + \frac{\abs{S_{1i\ell}}}{2}, ~~ s_i + \sum_{(i,n,j)\in S_{1i\ell}}
\frac{(y_{inj}-\mus_{1\ell})^2}{2}
}.
\]

\subsection{\texorpdfstring{Full Conditional for
\(s_i\)}{Full Conditional for s\_i}}\label{full-conditional-for-s_i}

\begin{align*}
p(s_i \mid \y, \rest) &\propto p(s_i) \times \prod_{z=0}^1 \prod_{\ell=1}^{L^z} p(\sss_{zi\ell} \mid s_i)\\
&\propto s_i^{a_s-1} \exp\bc{-b_s s_i} \times \prod_{z=0}^1  \prod_{\ell=1}^{L^z} s_i^{a_\sigma} \exp\bc{-s_i / \sss_{zi\ell}} \\
&\propto s_i^{a_s + (L^0 + L^1)a_\sigma - 1} \exp\bc{-s_i \p{b_s + \sum_{z=0}^1 \sum_{\ell=1}^{L^z} 1 / \sss_{zi\ell}}}.
\end{align*}

\[
\therefore s_i \mid \y, \rest \sim 
\G\p{a_s + (L^0 + L^1)a_\sigma, ~~ b_s + \sum_{z=0}^1 \sum_{\ell=1}^{L^z} \frac{1}{\sss_{zi\ell}} }.
\]

\subsection{\texorpdfstring{Full Conditional for
\(\bm\gamma\)}{Full Conditional for \textbackslash{}bm\textbackslash{}gamma}}\label{full-conditional-for-bmgamma}

The prior for \(\gamma_{inj}\) is
\(p(\gamma_{inj} = \ell \mid Z_{j\lin}=z, \eta^z_{ij\ell}) = \eta^z_{ij\ell}\),
where \(\ell \in \bc{1,...,L^z}\).

\begin{align*}
p(\gamma_{inj}=\ell \mid \y, Z_{j\lin}=z, \rest) &\propto p(\gamma_{inj}=\ell) \times p(y_{inj} \mid \gamma_{inj}=\ell, \rest) \\
&\propto p(\gamma_{inj}=\ell) \times p(y_{inj} \mid \mus_{z\ell}, \sss_{zi\ell}, \rest) \\
%
&\propto \eta^z_{ij\ell} \times \N(y_{inj} \mid \mus_{z\ell}, \sss_{zi\ell}) \\
&\propto \eta^z_{ij\ell} \times (\sss_{zi\ell})^{-1/2}
\exp\bc{-\frac{(y_{inj} - \mus_{z\ell})^2}{2\sss_{zi\ell}}} \\
\end{align*}

The normalizing constant is obtained by summing the last expression over
\(\ell = 1,...,L^z\). Moreover, since \(\ell\) is discrete, a Gibbs
update can be done on \(\gamma_{inj}\).

\subsection{\texorpdfstring{Full Conditional for
\(\bm\eta\)}{Full Conditional for \textbackslash{}bm\textbackslash{}eta}}\label{full-conditional-for-bmeta}

The prior for \(\bm\eta^z_{ij}\) is
\(\bm \eta^z_{ij} \sim \Dir_{L^z}(a_{\eta^z})\), for \(z\in\bc{0,1}\).
So the full conditional for \(\bm\eta^z_{ij}\) is:

\begin{align*}
p(\bm \eta^z_{ij} \mid \rest) \propto&~~ p(\bm{\eta}^z_{ij}) \times \prod_{n=1}^{N_i} p(\gamma_{inj} \mid \bm \eta^z_{ij})\\
\propto&~~ p(\bm \eta^z_{ij}) \times \prod_{n=1}^{N_i}\prod_{\ell=1}^{L^z} \p{\eta^z_{ij\ell}}^{\Ind{ \gamma_{inj}=\ell ~\cap~ Z_{j\lin=z}}}\\
%
\propto&~~ \prod_{\ell=1}^{L^z} \p{\eta^z_{ij\ell}}^{a_{\eta^z}/L^z-1} \times 
\prod_{n=1}^{N_i}\prod_{\ell=1}^{L^z} \p{\eta^z_{ij\ell}}^{\Ind{ \gamma_{inj}=\ell ~\cap~ Z_{j\lin=z}}}\\
\propto&~~ \prod_{\ell=1}^{L^z} \p{\eta^z_{ij\ell}}^{\p{a_{\eta^z} / L^z + \sum_{n=1}^{N_i} \Ind{ \gamma_{inj}=\ell ~\cap~ Z_{j\lin=z}}} - 1}\\
%
\end{align*}

Therefore, \[
\bm{\eta}^z_{ij} \mid \y,\rest ~\sim~ \Dir_{L^z}\p{a^*_1,...,a^*_{L^z}}
\] where
\(a^*_\ell = a_{\eta^z}/L^z+\sum_{n=1}^{N_i}\Ind{\gamma_{inj}=\ell ~\cap~ Z_{j\lin}=z}\).
Consequently, the full conditional for \(\bm{\eta}^z_{ij}\) can be
sampled from directly from a Dirichlet distribution of the form above.

\subsection{\texorpdfstring{Full Conditional for
\(\bm v\)}{Full Conditional for \textbackslash{}bm v}}\label{full-conditional-for-bm-v}

The prior distribution for \(v_k\) are
\(v_k \mid \alpha \ind \Be(\alpha/K, 1)\), for \(k = 1,...,K\). So,
\(p(v_k \mid \alpha) = \alpha v_k^{\alpha/K-1}\).

Let \(S = \bc{(i,n)\colon \lin = k}\).

\begin{align*}
p(v_k \mid \y, \rest) &\propto p(v_k) \prod_{j=1}^J\prod_{(i,n)\in S} p(\y \mid v_k, \rest) \\
&\propto (v_k)^{\alpha/K-1} \prod_{j=1}^J \prod_{(i,n)\in S}
\sum_{\ell=1}^{L^{Z_{jk}}} \eta^{Z_{jk}}_{ij\ell} \cdot
\N(y_{inj} \mid \mus_{Z_{jk}\ell}, \sss_{Z_{jk}i\ell})
\end{align*}

\mhLogitSpiel{v_k}{\xi}

Note also that \(\mus_{Z_{jk}\ell}\) and \(\sss_{Z_{jk}i\ell}\) are
functions of \(v_k\), and should be computed accordingly. Moreover, we
will only recompute the likelihood (in the metropolis acceptance ratio)
when \(Z_{jk}\) becomes different.

\subsection{\texorpdfstring{Full Conditional for
\(\alpha\)}{Full Conditional for \textbackslash{}alpha}}\label{full-conditional-for-alpha}

\begin{align*}
p(\alpha \mid \y, \rest) &\propto p(\alpha) \times \prod_{k=1}^K p(v_k \mid \alpha) \\
&\propto \alpha^{a_\alpha - 1} \exp\bc{-b_\alpha \alpha} \times \prod_{k=1}^K 
\alpha~v_k^{\alpha/K} \\
&\propto \alpha^{a_\alpha + K -1} \exp\bc{-\alpha\p{b_\alpha - 
\frac{\sum_{k=1}^K \log v_k}{K}}}
\end{align*}

\[
\therefore \alpha \mid \y, \rest \sim 
\G\p{a_\alpha + K,~ b_\alpha - \frac{\sum_{k=1}^K \log v_k}{K}}
\]

\subsection{\texorpdfstring{Full Conditional for
\(\bm H\)}{Full Conditional for \textbackslash{}bm H}}\label{full-conditional-for-bm-h}

The prior for \(\h_k\) is \(\h_k \sim \N_J(0, \Gamma)\). We can
analytically compute the conditional distribution
\(h_{j,k} \mid \h_{-j,k}\), which is

\[
h_{jk}  \mid \h_{-j,k} \sim \N(m_j, S^2_j),
\]

where

\[
\begin{cases}
m_j &= \bm G_{j,-j} \bm G_{-j,-j}^{-1}(\h_{-j,k})\\
S_j^2 &= \bm G_{j,j} - \bm G_{j,-j}\bm G_{-j,-j}^{-1}\bm G_{-j,j}\\
\end{cases}
\]

and the notation \(\h_{-j,k}\) refers to the vector \(h_k\) excluding
the \(j^{th}\) element. Likewise, \(\bm G_{-j,k}\) refers to the
\(k^{th}\) column of the matrix \(\bm G\) excluding the \(j^{th}\) row.

Note that if \(\bm G = \I_J\), then \(m_j=0\) and \(S_j^2 = 1\). Let
\(S = \bc{(i,n)\colon \lin=k}\).

\begin{align*}
p(h_{jk} \mid \y, \rest)  &\propto p(h_{jk}) \prod_{(i,n) \in S} p(y_{inj} \mid h_{jk}, \rest) \\
%
&\propto
\exp\bc{\frac{-(h_{jk} - m_j)^2}{2S_j^2}}
 \prod_{j=1}^J \prod_{(i,n)\in S}
\sum_{\ell=1}^{L^{Z_{jk}}} \eta^{Z_{jk}}_{ij\ell} \cdot
\N(y_{inj} \mid \mus_{Z_{jk}\ell}, \sss_{Z_{jk}i\ell})
\end{align*}

\mhSpiel{h_{jk}}

Note also that \(\mus_{Z_{jk}\ell}\) and \(\sss_{Z_{jk}i\ell}\) are
functions of \(h_{jk}\), and should be computed accordingly. Moreover,
we will only recompute the likelihood (in the metropolis acceptance
ratio) when \(Z_{jk}\) becomes different.  
\subsection{\texorpdfstring{Full Conditional for
\(\bm \lambda\)}{Full Conditional for \textbackslash{}bm \textbackslash{}lambda}}\label{full-conditional-for-bm-lambda}

The prior for \(\lin\) is \(p(\lin = k \mid \bm W_i) = W_{ik}\).

\begin{align*}
p(\lin=k\mid \y,\rest) &\propto p(\lin=k) ~ p(\y \mid \lin=k, \rest) \\
&\propto W_{ik}
\prod_{j=1}^J 
\p{
  \sum_{\ell=1}^{L^{Z_{jk}}} \eta^{Z_{jk}}_{ij\ell} \cdot
  \N(y_{inj} \mid 
  \mus_{Z_{jk}\ell}, \sss_{Z_{jk}i\ell})
}\\
\end{align*}

The normalizing constant is obtained by summing the last expression over
\(k = 1,...,K\). Moreover, since \(k\) is discrete, a Gibbs update can
be done on \(\lin\).

\subsection{\texorpdfstring{Full Conditional for
\(\bm W\)}{Full Conditional for \textbackslash{}bm W}}\label{full-conditional-for-bm-w}

The prior for \(\bm{W}_i\) is \(\bm W_i \sim \Dir(d, \cdots, d)\). So
the full conditional for \(\bm{W}_i\) is:

\begin{align*}
p(\bm W_i \mid \rest) \propto&~~ p(\bm{W}_i) \times \prod_{n=1}^{N_i} p(\lin \mid \bm{W}_i)\\
\propto&~~ p(\bm{W}_i) \times \prod_{n=1}^{N_i}\prod_{k=1}^K W_{ik}^{\Ind{\lin=k}}\\
\propto&~~ \prod_{k=1}^K W_{ik}^{d/K-1} \times \prod_{n=1}^{N_i}\prod_{k=1}^K W_{ik}^{\Ind{\lin=k}}\\
\propto&~~ \prod_{k=1}^K W_{ik}^{\p{d/K + \sum_{n=1}^{N_i}\Ind{\lin=k}}-1}\\
%
\end{align*}

Therefore, \[
\bm{W}_i \mid \y,\rest ~\sim~ \Dir\p{d/K+\sum_{n=1}^{N_i}\Ind{\lambda_{i,n}=1},...,d/K+\sum_{n=1}^{N_i}\Ind{\lambda_{i,n}=K}} 
\]

Consequently, the full conditional for \(\bm{W}_i\) can be sampled from
directly from a Dirichlet distribution of the form above.



\end{document}
