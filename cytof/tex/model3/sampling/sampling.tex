\documentclass[12pt]{article}
%{{{1
%\usepackage{lmodern}
\usepackage{amssymb,amsmath}
\usepackage{ifxetex,ifluatex}
\usepackage{fixltx2e} % provides \textsubscript
\ifnum 0\ifxetex 1\fi\ifluatex 1\fi=0 % if pdftex
  \usepackage[T1]{fontenc}
  \usepackage[utf8]{inputenc}
\else % if luatex or xelatex
  \ifxetex
    \usepackage{mathspec}
  \else
    \usepackage{fontspec}
  \fi
  \defaultfontfeatures{Ligatures=TeX,Scale=MatchLowercase}
\fi
% use upquote if available, for straight quotes in verbatim environments
\IfFileExists{upquote.sty}{\usepackage{upquote}}{}
% use microtype if available
\IfFileExists{microtype.sty}{%
\usepackage{microtype}
\UseMicrotypeSet[protrusion]{basicmath} % disable protrusion for tt fonts
}{}
\usepackage[margin=1in]{geometry}
\usepackage[unicode=true]{hyperref}
\hypersetup{
            pdftitle={Sampling Scheme for CyTOF Model3},
            pdfauthor={Arthur Lui},
            pdfborder={0 0 0},
            breaklinks=true}
\urlstyle{same}  % don't use monospace font for urls
\usepackage{natbib}
\bibliographystyle{plainnat}
\IfFileExists{parskip.sty}{%
\usepackage{parskip}
}{% else
\setlength{\parindent}{0pt}
\setlength{\parskip}{6pt plus 2pt minus 1pt}
}
\setlength{\emergencystretch}{3em}  % prevent overfull lines
\providecommand{\tightlist}{%
  \setlength{\itemsep}{0pt}\setlength{\parskip}{0pt}}
\setcounter{secnumdepth}{5}
% Redefines (sub)paragraphs to behave more like sections
\ifx\paragraph\undefined\else
\let\oldparagraph\paragraph
\renewcommand{\paragraph}[1]{\oldparagraph{#1}\mbox{}}
\fi
\ifx\subparagraph\undefined\else
\let\oldsubparagraph\subparagraph
\renewcommand{\subparagraph}[1]{\oldsubparagraph{#1}\mbox{}}
\fi
\usepackage{bm}
\usepackage{bbm}
\usepackage{graphicx}
\newcommand{\norm}[1]{\left\lVert#1\right\rVert}
\newcommand{\p}[1]{\left(#1\right)}
\newcommand{\bk}[1]{\left[#1\right]}
\newcommand{\bc}[1]{ \left\{#1\right\} }
\newcommand{\abs}[1]{ \left|#1\right| }
\newcommand{\mat}{ \begin{pmatrix} }
\newcommand{\tam}{ \end{pmatrix} }
\newcommand{\suml}{ \sum_{i=1}^n }
\newcommand{\prodl}{ \prod_{i=1}^n }
\newcommand{\ds}{ \displaystyle }
\newcommand{\df}[2]{ \frac{d#1}{d#2} }
\newcommand{\ddf}[2]{ \frac{d^2#1}{d{#2}^2} }
\newcommand{\pd}[2]{ \frac{\partial#1}{\partial#2} }
\newcommand{\pdd}[2]{\frac{\partial^2#1}{\partial{#2}^2} }
\newcommand{\N}{ \mathcal{N} }
\newcommand{\E}{ \text{E} }
\def\given{~\bigg|~}
\usepackage{float}
\def\beginmyfig{\begin{figure}[H]\center}
\def\endmyfig{\end{figure}}
\newcommand{\iid}{\overset{iid}{\sim}}
\newcommand{\ind}{\overset{ind}{\sim}}
\newcommand{\I}{\mathrm{\mathbf{I}}}
\allowdisplaybreaks
\def\M{\mathcal{M}}
\def\logit{\text{logit}}
\def\Bern{\text{Bernoulli}}
\def\N{\text{Normal}}
\def\G{\text{Gamma}}
\def\IG{\text{Inverse-Gamma}}
\def\Dir{\text{Dirichlet}}
\def\Be{\text{Beta}}
\def\lin{\lambda_{in}}
\def\btheta{\bm{\theta}}
\def\y{\bm{y}}
\newcommand\m{\bm{m}}
\def\mus{\mu^*}
\def\sss{{\sigma^2}^*}
\newcommand{\mhSpiel}[1]{
Since the full conditional distribution cannot be directly sampled from, it may
be sampled from by a Metropolis step with a Normal proposal distribution. 
The proposed state is accepted with probability 
$$
\min\bc{1, \frac{p(\tilde #1 \mid \y, \rest)}{p(#1\mid \y, \rest)}}.
$$
}


\newcommand{\mhLogSpiel}[2]{
Since the full conditional distribution for $#1$ cannot be directly sampled
from, it may be sampled from by a Metropolis step with a Normal proposal
distribution.  The parameter first needs to be log-transformed.  
Let the full conditional of the transformed parameter be 
$p(#2 \mid \y, \rest) = p_{#1}(\exp(#2) \mid \y, \rest) \exp(#2)$.
Then, the proposed state of the transformed parameter ($#2$) is accepted with
probability
$$
\min\bc{1, 
  \frac{p(\tilde #2  \mid \y, \rest)}
       {p(#2 \mid \y, \rest)}
}.
$$
Exponentiating the updated value for $#2$ returns the updated value for $#1$.
}


\newcommand{\mhLogitSpiel}[2]{
Since the full conditional distribution for $#1$ cannot be directly sampled
from, it may be sampled from by a Metropolis step with a Normal proposal
distribution.  The parameter first needs to be logit-transformed. 
Let the full conditional of the transformed parameter be 
$p(#2 \mid \y, \rest) = p_{#1}\p{\frac{1}{1+\exp(-#2)} \mid \y, \rest} 
\frac{\exp(-#2)}{(1 + \exp(- #2))^2}$.
The proposed
state of the transformed parameter ($#2$) is accepted with probability 
$$
\min\bc{1, 
  \frac{p(\tilde #2  \mid \y, \rest)}
       {p(#2 \mid \y, \rest)}
}
$$
Taking the inverse-logit of the updated value for $#2$ returns the updated
value for $#1$.
}



\newcommand{\Ind}[1]{\mathbbm{1}\bc{#1}}
\def\rest{\text{rest}}
\def\bang{\boldsymbol{\cdot}}
\def\h{\bm{h}}
\def\Z{\bm{Z}}
\def\W{\bm{W}}
\def\Unif{\text{Unif}}

\title{Sampling Scheme for CyTOF Model3}
\author{Arthur Lui}
\date{\today}



%%% Begin: Juhee
\newcommand{\indep}{\stackrel{indep}{\sim}}

\usepackage[dvipsnames,usenames]{color}
\newcommand{\ech}{\color{Black}\rm}
\newcommand{\hh}{\color{Mahogany}\it}

\usepackage{setspace}
%%% End: Juhee
%}}}1


\begin{document}
\onehalfspacing

\maketitle

\section{Notation}\label{notation}

Samples are taken from $I$ subjects, $i = 1,2,...,I$. Sample $i$ consists of
$N_i$ cells, and for each cell, expression levels of $J$ markers are measured.
Let a $J$-dimensional vector $\y_{in}=(y_{in1}, \ldots, y_{inJ})^\prime$, $i=1,
\ldots, I$ and $n=1, \ldots, N_i$ denote the measurements of expression levels
of $J$ markers for cell $n$ in sample $i$ where element $\tilde{y}_{inj} \in
\mathbb{R}^+$ represents the raw measurement of an expression level of marker
$j$ of cell $n$ in sample $i$. Let $c_{ij}$ denote the ``cutoff'' values
for sample $i$, marker $j$.
%{\tt explain ``cutoff values'' in one sentence.}
The cutoff values are obtained as part of the CyTOF analysis. Each marker and
sample comes with a cutoff value. Beyond the cutoff, ``positive'' expression
for a marker starts. In practice, some noise is detected from the environment
when the samples are measured. Hence, a set of cutoffs are recorded for each
marker for each sample as well. But this variation should be small.
%
We consider the logarithm transformation after
scaling $\tilde{y}_{inj}$ by $c_{ij}$, $$
y_{inj}=\log\p{\frac{\tilde{y}_{inj}}{c_{ij}}} \in \mathbb{R}.  $$ For some
$(i, n, j)$, $\tilde{y}_{inj}$ is missing.
%{\tt explain why missing in a sentence}
%Muftuoglu, Muharrem:
%That's what we expect and see based on how the software works when it converts
%data to fcs files. Negative events pile up at zero. 
%When the cell do not express a given marker and crosstalk into that channel is
%minimal we could see those zero values . 
Expression levels are recorded as missing when the markers are not expressed
and background signals for the marker are minimal. 
%
To account for missing data, we introduce a binary indicator, $$
m_{inj} = \begin{cases}
  0, & \text{if $\tilde{y}_{inj}$ is observed,} \\
  1, & \text{if $\tilde{y}_{inj}$ is missing.}
\end{cases}
$$ That is, $m_{inj}=1$ indicates that the expression level of marker
$j$ of cell $n$ in sample $i$ is not observed. 

\section{Sampling Model}\label{sampling-model}

We first assume $K$ latent cell phenotypes and a cell takes one of the
phenotypes. Each phenotype is defined based on a combination of expression or
no expression of $J$ markers in a cell. We let a $J \times K$ binary matrix
$\Z$ characterize the latent phenotypes, a column for a phenotype, where values
0 and 1 of $z_{jk}$ represent no expression and expression of marker $j$ in
cell phenotype $k$.  We introduce latent indicators for latent cell phenotypes
of cell $n$ in sample $i$, $\lambda_{in}$, $i=1, \ldots, I$ and $n=1, \ldots,
N_i$. $\lambda_{in} \in \{1, \ldots, K\}$ denotes the phenotype of cell $n$.
% in sample $i$ defined by column $k$ of . Element $z_{j, k} \in \{0, 1\}$
% indicates if marker $j$ is expressed in cell phenotype $k$. Event $z_{j,k}=0$
% represents that marker $j$ is not expressed, and $z_{j,k}=1$ for expression. 
Given $z_{j, \lambda_{i,n}}$, we assume a mixture of normals for $y_{inj}$,
%
\begin{align*}
y_{inj} \mid \mu^*, \sigma^{2 *}_{i} \ind
\begin{cases}
\sum_{\ell=1}^{L^0} \eta^0_{ij\ell} \N(\mu^\star_{0\ell}, \sigma^{2 \star}_{0i\ell}), &\mbox{if $z_{j,k}=0$},\\
\sum_{\ell=1}^{L^1} \eta^1_{ij\ell} \N(\mu^\star_{1\ell}, \sigma^{2 \star}_{1i\ell}), &\mbox{if $z_{j,k}=1$},\\
\end{cases}
\end{align*}
%
where $L^0$ and $L^1$ are fixed, $0 < \eta^0_{ij\ell} < 1$ with
$\sum_{\ell} \eta^0_{ij\ell}=1$ and $0 < \eta^1_{ij\ell} < 1$ with
$\sum_{\ell} \eta^1_{ij\ell}=1$. Given $\lambda_{in}=k$ we define
$\gamma_{inj}$; for $i=1, \ldots, I$, $n=1, \ldots, N_i$ and
$j=1, \ldots, J$,
%
\begin{eqnarray*}
p(\gamma_{inj} = \ell)=\eta^{z_{jk}}_{in\ell}, \mbox{ where }~ \ell \in \{1,\ldots, L^{z_{jk}}\}.
\end{eqnarray*}
%
Given $\lambda_{in}=k$ and $\gamma_{inj}=\ell$, we assume a normal
distribution for $y_{inj}$; for $i=1, \ldots, I$,
$n=1, \ldots, N_i$ and $j=1, \ldots, J$,
%
\begin{align*}
  y_{inj} \mid \mu_{inj}, \sigma^2_{inj}  &\ind \N(\mu_{inj}, \sigma^2_{inj}),
\end{align*}
%
where $\mu_{inj} = \mu^\star_{z_{j,k},\ell}$ and
$\sigma^2_{inj} = {\sigma^{2}}^\star_{iz_{j,k}\ell}$. Given
$y_{inj}$, we assume a Bernoulli distribution for $m_{inj}$; for
$i=1, \ldots, I$, $n=1, \ldots, N_i$ and $j=1, \ldots, J$,
%
\begin{align*}
  m_{inj} \mid p_{inj} &\ind \Bern(p_{inj}) \\
  \logit(p_{inj}) &:= \begin{cases}
  \beta_{0i} - \beta_{1i}(y_{inj}-c_0)^2, & \text{if } y_{inj} < c_0\nonumber, \\
  \beta_{0i} - \beta_{1i}c_1\sqrt{y_{inj}-c_0}, & \text{otherwise}, \nonumber \\
  \end{cases}
\end{align*}
where $c_0$ and $c_1$ are fixed.

Let $\btheta$ represent all parameters (discussed in the next
section). Let $\y$ represent all $y_{inj} ~ \forall(i,n,j)$. Let
$\m$ represent all $m_{inj} ~ \forall(i,n,j)$. The resulting \textbf{likelihood} is as follows:
\begin{align}
\mathcal{L} = p(\y, \m \mid \btheta) &= p(\m \mid \y) p(\y \mid \btheta) \nonumber\\
&= \prod_{i,n,j} p(m_{inj} \mid y_{inj}, \btheta) p(y_{inj} \mid \btheta) \nonumber\\
&= \prod_{i,n,j} \bc{
  p_{inj}^{m_{inj}} (1-p_{inj})^{1-m_{inj}} \times 
   \frac{1}{\sqrt{2\pi\sigma^2_{inj}}} \exp\bc{-\frac{(y_{inj}-\mu_{inj})^2}{2\sigma^2_{inj}}}
}.
\end{align}
The model is fully specified after priors are placed on all unknown
parameters. The marginal density for $y_{i,n,j}$ after integrating out
$\lambda$ and $\gamma$ is

\begin{align}
p(y_{inj} \mid \theta) = \sum_{k=1}^K W_{ik} \sum_{\ell=1}^{L^{Z_{jk}}}
\eta^{Z_{jk}}_{ij\ell} \cdot \N(y_{inj} \mid \mu^*_{Z_{jk}, \ell}, {\sigma^2}^*_{i,Z_{jk},\ell}).
\end{align}

\section{Priors}\label{priors}

Let $\bm\theta$ be \ldots{} \textbf{TODO}

{\bf Latent Phenotype} for $J\times K$ binary matrix $\Z$,
\begin{align*}
v_k \mid \alpha &\iid \Be(\alpha/K, 1),~ k=1, \ldots, K, \\
\alpha &\sim \G(a_\alpha, b_\alpha) \\
\h_k &\iid \N_J(\bm{0}, \bm G) \\
Z_{jk} \mid h_{jk}, v_k &:=
\Ind{\Phi(h_{jk} \mid 0, \bm G_{jj}) < v_k} 
\end{align*}
%{\tt explain here about IBP and the relationship between ours and IBP}.
Our main inferential objective is to obtain a binary matrix $\Z$ which contains
the latent phenotypes present in the samples. The number of phenotypes is not
known beforehand. A suitable prior for the latent phenotypes is an Indian
buffet process (IBP), a distribution over binary matrices of infinite
dimensions. When $K$ is taken to the limit and $\bm G$ is the identity matrix,
the representation yields the traditional IBP. Our model additionally accounts
for the correlation between markers through $\bm G$.

We represent a sample as an admixture of heterogeneous cells having $K$ different latent cell phenotypes characterized by $\Z$. A $K-$dimensional vector $\W_i$ denotes the proportion of a mixture for sample $i$ and we assume
\begin{align*}
&\W_{i} \iid \Dir_K(d/K, \ldots, d/K),\\
&p(\lin=k \mid \bm \W_i) = W_{ik}.
\end{align*}
Recall that the likelihood for $y_{inj}$ is a mixture of normals with means $\eta_{0\ell}$, $\ell=1, \ldots, L^0$ and $\eta_{1\ell}$, $\ell=1, \ldots, L^1$ conditional on $z_{j \lambda_{in}}=0$ and 1, respectively. Given that maker $j$ is expressed in phenotype $k$, $z_{jk}=1$, cells taking phenotype $k$ may have a positive mean for the observed expression level of the marker.  Similarly, for $z_{jk}=0$ the mean of $y_{inj}$ for cells taking phenotype $k$ may take a negative value.  We thus assume
\begin{align*}
\mus_{0\ell} \mid \psi_0, \tau^2_0 &\iid \N_-(\psi_0, \tau^2_0), ~~~ \ell \in \bc{1,...,L^0}, \\
\mus_{1\ell} \mid \psi_1, \tau^2_1 &\iid \N_+(\psi_1, \tau^2_1), ~~~ \ell \in \bc{1,...,L^1}. 
\end{align*}
Note that $X ~ \N_-(m,s^2)$ denotes that $X$ is distributed Normally
with mean $m$ and variance $s^2$, truncated to take on only
\textbf{negative} values. Similarly, $X ~ \N_+(m,s^2)$ denotes that
$X$ is distributed Normally with mean $m$ and variance $s^2$,
truncated to take on only \textbf{positive} values.

For $i=1, \ldots, I$,
%
\begin{align*}
\sigma^2_{0i\ell} \mid s_i &\ind \IG(a_\sigma, s_i), ~~~ \ell \in \bc{1,...,L^0}, \\
\sigma^2_{1i\ell} \mid s_i &\ind \IG(a_\sigma, s_i), ~~~ \ell \in \bc{1,...,L^1}, \\
s_i &\iid \G(a_s, b_s), ~~~ i \in \bc{1,...,I}.
\end{align*}

For $i=1, \ldots, I$, $n=1, \ldots, N_i$ and $j=1, \ldots, J$,
%
\begin{align*}
p(\gamma_{inj} = \ell \mid \bm\eta^{Z_{j \lambda_{in}}}_{i,j}) &= \eta^{Z_{j \lambda_{in}}}_{ijl},
~~~ \ell \in \bc{1,...,L^{Z_{j \lambda_{in}}}},  \\
\bm\eta^0_{ij} &\iid \Dir_{L^0}(a_{\eta^0}/L^0, \ldots, a_{\eta^0}/L^0), \\
\bm\eta^1_{ij} &\iid \Dir_{L^1}(a_{\eta^1}/L^1, \ldots, a_{\eta^1}/L^1). 
\end{align*}


For $i=1, \ldots, I$,
%
\begin{align*}
\beta_{0i} &\iid \N(m_{\beta_0}, s^2_{\beta_0}) ~~~
\text{(hyper-parameters determined empirically)}, \\
\beta_{1i} &\iid \G(a_{\beta_1}, b_{\beta_1}) ~~~~
\text{(with mean $a_{\beta_1}/b_{\beta_1}$, determined empirically)}. 
\end{align*}
%
The Gamma distribution with parameters $(a,b)$ has mean $a/b$. The
inverse-Gamma distribution with parameters $(a,b)$ has mean
$b / (a-1)$. For the mixture locations of the likelihood,
%


\section{Sampling via MCMC}\label{sampling-via-mcmc}

Sampling can be done via Gibbs sampling by repeatedly updating each
parameter one at a time until convergence. Parameter updates are made by
sampling from it full conditional distribution. Where this cannot be
done conveniently, a metropolis step will be necessary.

To sample from a distribution which is otherwise difficult to sample
from, the Metropolis-Hastings algorithm can be used. This is
particularly useful when sampling from a full conditional distribution
of one of many parameters in an MCMC based sampling scheme (such as a
Gibbs sampler). Say $B$ samples from a distribution with density
$p(\theta)$ is desired, one can do the following:

\begin{enumerate}
\def\labelenumi{\arabic{enumi}.}
\tightlist
\item
  Provide an initial value for the sampler, e.g. $\theta^{(0)}$.
\item
  Repeat the following steps for $i = 1,...,B$.
\item
  Sample a new value $\tilde\theta$ for $\theta^{(i)}$ from a
  proposal distribution $Q(\cdot \mid \theta^{(i-1)})$.

  \begin{itemize}
  \tightlist
  \item
    Let $q(\tilde\theta \mid \theta)$ be the density of the proposal
    distribution.
  \end{itemize}
\item
  Compute the ``acceptance ratio'' to be $$
     \rho=
     \min\bc{1, \frac{p(\tilde\theta)}{p(\theta^{(i-1)})} \Big/ 
            \frac{q(\tilde\theta\mid\theta^{(i-1)})}
                 {q(\theta^{(i-1)}\mid\tilde\theta)}
        }
     $$
\item
  Set $$
     \theta^{(i)} := 
     \begin{cases}
     \tilde\theta &\text{with probability } \rho \\
     \theta^{(i-1)} &\text{with probability } 1-\rho.
     \end{cases}
     $$
\end{enumerate}

Note that in the case of a symmetric proposal distribution, the
acceptance ratio simplifies further to be
$\frac{p(\tilde\theta)}{p(\theta^{(i-1)})}$.

The proposal distribution should be chosen to have the same support as
the parameter. Transforming parameters to have infinite support can,
therefore, be convenient as a Normal proposal distribution can be used.
Moreover, as previously mentioned, the use of symmetric proposal
distributions (such as the Normal distribution) can simplify the
computation of the acceptance ratio.

Some common parameter transformations are therefore presented here:

\begin{enumerate}
\def\labelenumi{\arabic{enumi}.}
\tightlist
\item
  For parameters bounded between $(0,1)$, a logit-transformation may
  be used. Specifically, if a random variable $X$ with density
  $f_X(x)$ has support in the unit interval, then
  $Y=\logit(X)=\log\p{\frac{p}{1-p}}$ will have density
  $f_Y(y) = f_X\p{\frac{1}{1+\exp(-y)}}\frac{e^{-y}}{(1+e^{-y})^{2}}$.
\item
  For parameters with support $(0,\infty)$, a log-transformation may
  be used. Specifically, if a random variable $X$ with density
  $f_X(x)$ has positive support, then $Y = \log(X)$ has pdf
  $f_Y(y) = f_X(e^y) e^y$.
\end{enumerate}

\section{Full Conditionals}\label{full-conditionals}

\subsection{\texorpdfstring{Full Conditional for
$\bm \beta$}{Full Conditional for \textbackslash{}bm \textbackslash{}beta}}\label{full-conditional-for-bm-beta}

Define $f_{inj}$ to be
%
\begin{align*}
f_{inj} &:= P(m_{inj} \mid p_{inj}, y_{inj}) \\
&= p_{inj}^{m_{inj}} (1-p_{inj})^{1 - m_{inj}} \\
&= \left(\frac{1}{1+e^{-x_{inj}}} \right)^{m_{inj}}\left(\frac{1}{1+e^{x_{inj}}} \right)^{1-m_{inj}},
\end{align*}
%
where
\begin{align*}
  x_{inj} &:= \begin{cases}
  \beta_{0i} - \beta_{1i}(y_{inj}-c_0)^2, & \text{if } y_{inj} < c_0\nonumber \\
  \beta_{0i} - \beta_{1i}c_1\sqrt{y_{inj}-c_0}, & \text{otherwise}, \nonumber \\
  \end{cases}\\
\end{align*}
where $c_0$ and $c_1$ are fixed.

%TODO
%For convenience in specifying priors for $\beta_0i$, $\beta_1i$, and $c_1$ in the missing
%mechanism, we predetermine 

\textbf{Write a little about the following...} \\
Predetermine $c_\text{low}$, $c_\text{high}$, $p_\text{low}$, $p_\text{high}$, and
$p_0$, which will determine the shape of the missing mechanism. Then

\begin{align*}
  \beta_0 &:= \logit(p_0) \\
  \beta_1 &:= \frac{\beta_0 - \logit(p_\text{low})}{(y_\text{low} - c_0)^2} \\
  c_1 &:= \frac{\beta_0 - \logit(p_\text{high})}{\beta_1 ~ \sqrt{y_\text{high} - c_0} }. \\
\end{align*}


\subsubsection{\texorpdfstring{Full Conditional for
$\beta_{0i}$}{Full Conditional for \textbackslash{}beta\_\{0i\}}}\label{full-conditional-for-beta_0i}

Recall that $\beta_{0i} \iid \N(m_{\beta_0},s^2_{\beta_0})$.
\begin{align*}
p(\beta_{0i} \mid \y, \rest) &\propto
p(\beta_{0i}) \times \prod_{n=1}^{N_i} \prod_{j=1}^J f_{inj} \\
%
&\propto \exp\bc{\frac{-(\beta_{0i}-m_{\beta_0})^2}{2s^2_{\beta_0}}} \prod_{n=1}^{N_i} \prod_{j=1}^J f_{inj} \\
\end{align*}

\mhSpiel{\beta_{0i}}

\subsubsection{\texorpdfstring{Full Conditional for
$\beta_{1i}$}{Full Conditional for \textbackslash{}beta\_\{1i\}}}\label{full-conditional-for-beta_1i}

Recall that $\beta_{1i}\ind \G(a_{\beta_1}, b_{\beta_1})$.
%
\begin{align*}
p(\beta_{1i} \mid \y, \rest) &\propto
p(\beta_{1i}) \times 
\prod_{n=1}^{N_i} \prod_{j=1}^J f_{inj} \\
%
&\propto \beta_{1i}^{a_{\beta_1}-1}\exp\bc{-b_{\beta_1}\beta_{1i}} 
\prod_{n=1}^{N_i} \prod_{j=1}^J f_{inj} \\
\end{align*}
%
\mhLogSpiel{\beta_{1i}}{\xi}

\subsection{\texorpdfstring{Full Conditional for Missing
$\bm \y$}{Full Conditional for Missing \textbackslash{}bm \textbackslash{}y}}\label{full-conditional-for-missing-bm-y}
%
\begin{align*}
p(y_{inj} \mid m_{inj}, \rest) &\propto
p(y_{inj} \mid \rest) ~
p(m_{inj} \mid y_{inj}, \rest) \\
%
&\propto
f_{inj} 
\sum_{\ell=1}^{L^{Z_{j\lin}}} \eta^{Z_{j\lin}}_{ij\ell} \cdot \N(y_{inj} \mid \mu^*_{Z_{jk}, \ell}, {\sigma^2}^*_{i,Z_{j\lin},\ell}).
\end{align*}
%
\mhSpiel{y_{inj}}
%
Note that $f_{inj}$ is a function of $y_{inj}$ and should be
computed accordingly.

\subsection{\texorpdfstring{Full Conditional for
$\bm\mu^\star$}{Full Conditional for \textbackslash{}bm\textbackslash{}mu\^{}\star}}\label{full-conditional-for-bmmu}

For $\mus_{0\ell}$, let
$S_{0i\ell} = \bc{(i,n,j) : \p{Z_{j,\lin} = 0 ~\cap~ \gamma_{inj} = \ell}}$g
and $|S_{0i\ell}|$ the cardinality of $S_{0i\ell}$.
%
\newcommand\musZeroPostvarDenom{
  \frac{1}{\tau^2_0} + \sum_{i=1}^I\frac{|S_{0i\ell}|}{{\sigma^2}^\star_{0i\ell}}
}
\newcommand\musZeroPostMeanNum{
  \frac{\psi_0}{\tau^2_0} + 
  \sum_{i=1}^I \sum_{S_{0i\ell}}  
  \frac{y_{inj}}{{\sigma^2}^*_{0i\ell}}
}
%
\begin{align*}
p(\mus_{0\ell} \mid \y, \rest) &\propto 
p(\mus_{0\ell} \mid \psi_0, \tau^2_0) \times p(\y \mid \mus_{0\ell},\rest) \\
%
&\propto
\Ind{\mus_{0\ell}<0} \exp\bc{\frac{-(\mus_{0\ell} - \psi_0)^2}{2\tau^2_{0}}}
\prod_{i=1}^I\prod_{(i,n,j)\in S_{0i\ell}} \exp\bc{\frac{-(y_{inj} - \mus_{0\ell})^2}{2{\sigma^2}^\star_{i0\ell}}} \\
%
&\propto
\exp\bc{
  -\frac{(\mus_{0\ell})^2}{2}\p{\musZeroPostvarDenom} + 
  \mus_{0\ell}\p{\musZeroPostMeanNum}
} \\ 
& ~~~ \times \Ind{\mus_{0i\ell}<0} \\
\end{align*}
%
$$
\renewcommand\musZeroPostvarDenom{
  1 + \tau^2_0\sum_{i=1}^I(|S_{0i\ell}|/{\sigma^2}^\star_{0i\ell})
}
\renewcommand\musZeroPostMeanNum{
  \psi_0 + \tau^2_0 \sum_{i=1}^I\sum_{S_{0i\ell}} (y_{inj} / {\sigma^2}^\star_{0i\ell})
}
\therefore \mus_{0l} \mid \y, \rest \ind \N_-\p{
  \frac{\musZeroPostMeanNum}{\musZeroPostvarDenom},
  \frac{\tau^2_0}{\musZeroPostvarDenom}
}
$$
Similarly for $\mus_{1\ell}$, let
$S_{1i\ell} = \bc{(i,n,j) : \p{Z_{j,\lin} = 1 ~\cap~ \gamma_{inj} = \ell}}$g
and $|S_{1i\ell}|$ the cardinality of $S_{1i\ell}$.
$$
\newcommand\musOnePostvarDenom{
  1 + \tau^2_1 \sum_{i=1}^I (|S_{1i\ell}|/{\sigma^2}^\star_{1i\ell})
}
\newcommand\musOnePostMeanNum{
  \psi_1 + \tau^2_1 \sum_{i=1}^I \sum_{S_{1i\ell}} (y_{inj} / {\sigma^2}^\star_{1i\ell})
}
\therefore \mus_{1l} \mid \y, \rest \ind \N_+\p{
  \frac{\musOnePostMeanNum}{\musOnePostvarDenom},
  \frac{\tau^2_1}{\musOnePostvarDenom}
}
$$

\subsection{\texorpdfstring{Full Conditional for
$\bm{{\sigma^2}}^*$}{Full Conditional for \textbackslash{}bm\{\{\textbackslash{}sigma\^{}2\}\}\^{}*}}\label{full-conditional-for-bmsigma2}

Let
$S_{0i\ell} = \bc{(i, n,j): Z_{j,\lin} = 0 ~\cap~ \gamma_{inj}=\ell}$,
$i=1, \ldots, I$.
\begin{align*}
p(\sss_{0i\ell} \mid \y, \rest) &\propto p(\sss_{0i\ell} \mid s_i) \times p(\y \mid \sss_{0i\ell}, \rest) \\
&\propto (\sss_{0i\ell})^{-a_\sigma-1} \exp\bc{-\frac{s_i}{\sss_{0i\ell}}} 
\prod_{(i,n,j)\in S_{0i\ell}} \bc{
  \frac{1}{\sqrt{2\sss_{0i\ell}}}
  \exp\bc{\frac{-(y_{inj}-\mus_{0\ell})^2}{2\sss_{0i\ell}}}
} \\
&\propto (\sss_{0i\ell})^{-(a_\sigma + \frac{\abs{S_{0i\ell}}}{2})-1}
\exp\bc{\p{\frac{1}{\sss_{0i\ell}}}\p{s_i + \sum_{(i,n,j)\in S_{0i\ell}}
\frac{(y_{inj}-\mus_{0\ell})^2}{2}
}}.
\end{align*}
%
$$
\therefore \sss_{0i\ell} \mid \y, \rest \ind
\IG\p{a_\sigma + \frac{\abs{S_{0i\ell}}}{2}, ~~ s_i + \sum_{(i,n,j)\in S_{0i\ell}}
\frac{(y_{inj}-\mus_{0\ell})^2}{2}
}.
$$
%
Similarly, let
$S_{1i\ell} = \bc{(i, n,j): Z_{j,\lin} = 1 ~\cap~ \gamma_{inj}=\ell}$.
Then, the full conditional for $\sss_{1i\ell}$ is $$
\therefore \sss_{1i\ell} \mid \y, \rest \ind
\IG\p{a_\sigma + \frac{\abs{S_{1i\ell}}}{2}, ~~ s_i + \sum_{(i,n,j)\in S_{1i\ell}}
\frac{(y_{inj}-\mus_{1\ell})^2}{2}
}.
$$

\subsection{\texorpdfstring{Full Conditional for
$s_i$}{Full Conditional for s\_i}}\label{full-conditional-for-s_i}
%
\begin{align*}
p(s_i \mid \y, \rest) &\propto p(s_i) \times \prod_{z=0}^1 \prod_{\ell=1}^{L^z} p(\sss_{zi\ell} \mid s_i)\\
&\propto s_i^{a_s-1} \exp\bc{-b_s s_i} \times \prod_{z=0}^1  \prod_{\ell=1}^{L^z} s_i^{a_\sigma} \exp\bc{-s_i / \sss_{zi\ell}} \\
&\propto s_i^{a_s + (L^0 + L^1)a_\sigma - 1} \exp\bc{-s_i \p{b_s + \sum_{z=0}^1 \sum_{\ell=1}^{L^z} 1 / \sss_{zi\ell}}}.
\end{align*}
%
$$
\therefore s_i \mid \y, \rest \sim 
\G\p{a_s + (L^0 + L^1)a_\sigma, ~~ b_s + \sum_{z=0}^1 \sum_{\ell=1}^{L^z} \frac{1}{\sss_{zi\ell}} }.
$$
%
\subsection{\texorpdfstring{Full Conditional for
$\bm\gamma$}{Full Conditional for \textbackslash{}bm\textbackslash{}gamma}}\label{full-conditional-for-bmgamma}

The prior for $\gamma_{inj}$ is
$p(\gamma_{inj} = \ell \mid Z_{j\lin}=z, \eta^z_{ij\ell}) = \eta^z_{ij\ell}$,
where $\ell \in \bc{1,...,L^z}$.
%
\begin{align*}
p(\gamma_{inj}=\ell \mid \y, Z_{j\lin}=z, \rest) &\propto p(\gamma_{inj}=\ell) \times p(y_{inj} \mid \gamma_{inj}=\ell, \rest) \\
&\propto p(\gamma_{inj}=\ell) \times p(y_{inj} \mid \mus_{z\ell}, \sss_{zi\ell}, \rest) \\
%
&\propto \eta^z_{ij\ell} \times \N(y_{inj} \mid \mus_{z\ell}, \sss_{zi\ell}) \\
&\propto \eta^z_{ij\ell} \times (\sss_{zi\ell})^{-1/2}
\exp\bc{-\frac{(y_{inj} - \mus_{z\ell})^2}{2\sss_{zi\ell}}} \\
\end{align*}
%
The normalizing constant is obtained by summing the last expression over
$\ell = 1,...,L^z$. Moreover, since $\ell$ is discrete, a Gibbs
update can be done on $\gamma_{inj}$.

\subsection{\texorpdfstring{Full Conditional for
$\bm\eta$}{Full Conditional for \textbackslash{}bm\textbackslash{}eta}}\label{full-conditional-for-bmeta}

The prior for $\bm\eta^z_{ij}$ is
$\bm \eta^z_{ij} \sim \Dir_{L^z}(a_{\eta^z} / L^z)$, for $z\in\bc{0,1}$.
So the full conditional for $\bm\eta^z_{ij}$ is:
%
\begin{align*}
p(\bm \eta^z_{ij} \mid \rest) \propto&~~ p(\bm{\eta}^z_{ij}) \times \prod_{n=1}^{N_i} p(\gamma_{inj} \mid \bm \eta^z_{ij})\\
\propto&~~ p(\bm \eta^z_{ij}) \times \prod_{n=1}^{N_i}\prod_{\ell=1}^{L^z} \p{\eta^z_{ij\ell}}^{\Ind{ \gamma_{inj}=\ell ~\cap~ Z_{j\lin}=z}}\\
%
\propto&~~ \prod_{\ell=1}^{L^z} \p{\eta^z_{ij\ell}}^{a_{\eta^z}/L^z-1} \times 
\prod_{n=1}^{N_i}\prod_{\ell=1}^{L^z} \p{\eta^z_{ij\ell}}^{\Ind{ \gamma_{inj}=\ell ~\cap~ Z_{j\lin}=z}}\\
\propto&~~ \prod_{\ell=1}^{L^z} \p{\eta^z_{ij\ell}}^{\p{a_{\eta^z}/L^z + \sum_{n=1}^{N_i} \Ind{ \gamma_{inj}=\ell ~\cap~ Z_{j\lin}=z}} - 1}\\
%
\end{align*}
%
Therefore, $$
\bm{\eta}^z_{ij} \mid \y,\rest ~\sim~ \Dir_{L^z}\p{a^*_1,...,a^*_{L^z}}
$$ where
$a^*_\ell = a_{\eta^z}/L^z+\sum_{n=1}^{N_i}\Ind{\gamma_{inj}=\ell ~\cap~ Z_{j\lin}=z}$.
Consequently, the full conditional for $\bm{\eta}^z_{ij}$ can be
sampled from directly from a Dirichlet distribution of the form above.

\subsection{\texorpdfstring{Full Conditional for
$\bm v$}{Full Conditional for \textbackslash{}bm v}}\label{full-conditional-for-bm-v}

The prior distribution for $v_k$ are
$v_k \mid \alpha \ind \Be(\alpha/K, 1)$, for $k = 1,...,K$. So,
$p(v_k \mid \alpha) = \alpha v_k^{\alpha/K-1}$.
%
Let $S = \bc{(i,n)\colon \lin = k}$.
%
\begin{align*}
p(v_k \mid \y, \rest) &\propto p(v_k) \prod_{j=1}^J\prod_{(i,n)\in S} p(\y \mid v_k, \rest) \\
&\propto (v_k)^{\alpha/K-1} \prod_{j=1}^J \prod_{(i,n)\in S}
\sum_{\ell=1}^{L^{Z_{jk}}} \eta^{Z_{jk}}_{ij\ell} \cdot
\N(y_{inj} \mid \mus_{Z_{jk}\ell}, \sss_{Z_{jk}i\ell})
\end{align*}
%
\mhLogitSpiel{v_k}{\xi}
%
Note also that $\mus_{Z_{jk}\ell}$ and $\sss_{Z_{jk}i\ell}$ are
functions of $v_k$, and should be computed accordingly. Moreover, we
will only recompute the likelihood (in the metropolis acceptance ratio)
when $Z_{jk}$ becomes different.

\subsection{\texorpdfstring{Full Conditional for
$\alpha$}{Full Conditional for \textbackslash{}alpha}}\label{full-conditional-for-alpha}

\begin{align*}
p(\alpha \mid \y, \rest) &\propto p(\alpha) \times \prod_{k=1}^K p(v_k \mid \alpha) \\
&\propto \alpha^{a_\alpha - 1} \exp\bc{-b_\alpha \alpha} \times \prod_{k=1}^K 
\alpha~v_k^{\alpha/K} \\
&\propto \alpha^{a_\alpha + K -1} \exp\bc{-\alpha\p{b_\alpha + 
\frac{\sum_{k=1}^K \log v_k}{K}}}
\end{align*}
%
$$
\therefore \alpha \mid \y, \rest \sim 
\G\p{a_\alpha + K,~ b_\alpha + \frac{\sum_{k=1}^K \log v_k}{K}}
$$

\subsection{\texorpdfstring{Full Conditional for
$\bm H$}{Full Conditional for \textbackslash{}bm H}}\label{full-conditional-for-bm-h}

The prior for $\h_k$ is $\h_k \sim \N_J(0, \bm G)$. We can
analytically compute the conditional distribution
$h_{j,k} \mid \h_{-j,k}$, which is
%
$$
h_{jk}  \mid \h_{-j,k} \sim \N(m_j, S^2_j),
$$
%
where
%
$$
\begin{cases}
m_j &= \bm G_{j,-j} \bm G_{-j,-j}^{-1}(\h_{-j,k})\\
S_j^2 &= \bm G_{j,j} - \bm G_{j,-j}\bm G_{-j,-j}^{-1}\bm G_{-j,j}\\
\end{cases}
$$
%
and the notation $\h_{-j,k}$ refers to the vector $h_k$ excluding
the $j^{th}$ element. Likewise, $\bm G_{-j,k}$ refers to the
$k^{th}$ column of the matrix $\bm G$ excluding the $j^{th}$ row.

Note that if $\bm G = \I_J$, then $m_j=0$ and $S_j^2 = 1$. Let
$S = \bc{(i,n)\colon \lin=k}$.

\begin{align*}
p(h_{jk} \mid \y, \rest)  &\propto p(h_{jk}) \prod_{(i,n) \in S} p(y_{inj} \mid h_{jk}, \rest) \\
%
&\propto
\exp\bc{\frac{-(h_{jk} - m_j)^2}{2S_j^2}}
 \prod_{j=1}^J \prod_{(i,n)\in S}
\sum_{\ell=1}^{L^{Z_{jk}}} \eta^{Z_{jk}}_{ij\ell} \cdot
\N(y_{inj} \mid \mus_{Z_{jk}\ell}, \sss_{Z_{jk}i\ell})
\end{align*}

\mhSpiel{h_{jk}}

Note also that $\mus_{Z_{jk}\ell}$ and $\sss_{Z_{jk}i\ell}$ are
functions of $h_{jk}$, and should be computed accordingly. Moreover,
we will only recompute the likelihood (in the metropolis acceptance
ratio) when $Z_{jk}$ becomes different.

\subsection{\texorpdfstring{Full Conditional for
$\bm \lambda$}{Full Conditional for \textbackslash{}bm \textbackslash{}lambda}}\label{full-conditional-for-bm-lambda}

The prior for $\lin$ is $p(\lin = k \mid \bm W_i) = W_{ik}$.
%
\begin{align*}
p(\lin=k\mid \y,\rest) &\propto p(\lin=k) ~ p(\y \mid \lin=k, \rest) \\
&\propto W_{ik}
\prod_{j=1}^J 
\p{
  \sum_{\ell=1}^{L^{Z_{jk}}} \eta^{Z_{jk}}_{ij\ell} \cdot
  \N(y_{inj} \mid 
  \mus_{Z_{jk}\ell}, \sss_{Z_{jk}i\ell}).
}
\end{align*}
%
The normalizing constant is obtained by summing the last expression over
$k = 1,...,K$. Moreover, since $k$ is discrete, a Gibbs update can
be done on $\lin$.

\subsection{\texorpdfstring{Full Conditional for
$\bm W$}{Full Conditional for \textbackslash{}bm W}}\label{full-conditional-for-bm-w}

The prior for $\bm{W}_i$ is $\bm W_i \sim \Dir(d/K, \cdots, d/K)$. So
the full conditional for $\bm{W}_i$ is:
%
\begin{align*}
p(\bm W_i \mid \rest) \propto&~~ p(\bm{W}_i) \times \prod_{n=1}^{N_i} p(\lin \mid \bm{W}_i)\\
\propto&~~ p(\bm{W}_i) \times \prod_{n=1}^{N_i}\prod_{k=1}^K W_{ik}^{\Ind{\lin=k}}\\
\propto&~~ \prod_{k=1}^K W_{ik}^{d/K-1} \times \prod_{n=1}^{N_i}\prod_{k=1}^K W_{ik}^{\Ind{\lin=k}}\\
\propto&~~ \prod_{k=1}^K W_{ik}^{\p{d/K + \sum_{n=1}^{N_i}\Ind{\lin=k}}-1}.
%
\end{align*}
%
Therefore, $$
\bm{W}_i \mid \y,\rest ~\sim~ \Dir\p{d/K+\sum_{n=1}^{N_i}\Ind{\lambda_{i,n}=1},...,d/K+\sum_{n=1}^{N_i}\Ind{\lambda_{i,n}=K}}. 
$$

Consequently, the full conditional for $\bm{W}_i$ can be sampled from
directly from a Dirichlet distribution of the form above.

\subsection{\texorpdfstring{Posterior Estimate for $Z$, $W$, and
$\lambda$}{Posterior Estimate for Z, W, and \textbackslash{}lambda}}\label{posterior-estimate-for-z-w-and-lambda}

Obtaining posterior estimates and quantifying uncertainty for $Z$,
$W$, and $\lambda$ can be a challenge since they depend directly on
$K$ and are susceptible to label-switching. Therefore, we describe a
way to select point-estimates for $Z$, $W$, and $\lambda$ from
their posterior samples.

Suppose we obtain $B$ samples from the posterior distribution of
$\theta$. Let $\theta^{(b)}$ denote sample $b$ in that sample for
$b \in \bc{1,...,B}$. Then, for each posterior sample of $Z$ and $W_i$ (i.e. $Z^{(b)}$
and $W_i^{(b)}$) let $A_i^{(b)}$ be a $J\times J$
adjacency-matrix;
$$
A^{(b)}_{i,j,j'} = \sum_{k=1}^K W^{(b)}_{i,k} \Ind{Z^{(b)}_{j,k} = 1}
\Ind{Z^{(b)}_{j',k} = 1}.
$$
%
Then, compute the mean adjacency-matrix as
$\bar A_i = \sum_{b=1}^B A_i^{(b)} / B$. The posterior estimate for
$\bm Z_i$ is then
%
$$
\hat{\bm Z}_i = \text{argmin}_{\bm Z} \sum_{j,j'} (A_{i,j,j'}^{(b)} - \bar A_{i,j,j'})^2).
$$
%
We search for such $\hat \Z_i$ using the Monte Carlo method. Suppose
$\hat{\bm Z}_i = \bm Z^{(b)}$. We then let
$\widehat{\bm W}_i = \bm W_i^{(b)}$, and
$\hat{\bm \lambda}_i = \bm \lambda_i^{(b)}$.

%\bibliography{../../bib/litreview.bib}

\end{document}
